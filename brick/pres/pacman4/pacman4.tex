\documentclass{beamer}
\usepackage[english,russian]{babel}
\usepackage[utf8]{inputenc}
% Стиль презентации
\usetheme{Warsaw}



\begin{document}
\title{Программные технологии для эффективного поиска логического вывода в исчислении позитивно-образованных формул}  
\author{Ларионов Александр Александрович}
\institute{Иркутский государственный университет}
%Научный руководитель: к.т.н. доцент Черкашин Е.А.
%Специальность: 05.13.17 Теоретические основы информатики
%Область исследования: 2,8
\date{Красноярск, 2012} 
% Создание заглавной страницы
\frame{\titlepage} 
% Автоматическая генерация содержания
\frame{\frametitle{Содержание}\tableofcontents} 


%=========================================================================================
%=======================Актуальность===================================
%=========================================================================================
\begin{frame}{Актуальность работы}
%Эффективность поиска ЛВ задается следующими критериями: время решения задачи; количество шагов ЛВ; объём использованной оперативной памяти; ширина класса успешно решаемых задач.
\end{frame}


%=========================================================================================
%=======================Объект и предмет===================================
%=========================================================================================
\begin{frame}{Объект и предмет исследования}

\begin{block}{Объект исследования}
Исчисление позитивно--образованных формул. 
\end{block}

\begin{block}{Предмет исследования}
Свойства языка и исчисления ПО--формул, современных структур данных и алгоритмов поиска ЛВ в исчислениях первого порядка, разработка структур данных и алгоритмов, обеспечивающих создание новых программных систем и технологий эффективного поиска ЛВ в исчислениях ПО--формул. 
\end{block}

\end{frame}

%=========================================================================================
%=======================Цель и задачи===================================
%=========================================================================================
\begin{frame}{Цель и задачи}

\begin{block}{Цель}
Разработка программной системы для эффективного поиска логического вывода в исчислении позитивно--образованных формул.
\end{block}

\begin{block}{Задачи}
\begin{enumerate}
\item Разработка эффективных структур данных представления ПО--формул в памяти компьютера.
\item Создание оригинальных методик (стратегий) эффективного поиска ЛВ в исчислении ПО--формул.
\item Адаптация существующих методик АДТ для исчисления ПО--формул с целью повышения эффективности поиска ЛВ.
\item Разработка эффективной программной системы АДТ, создание инструментальных средств для программирования систем АДТ в исчислении ПО--формул.
\item Апробация разработанных программных средств на решении тестовых и практических задач. Сравнение с другими системами АДТ.
\end{enumerate}
\end{block}

\end{frame}


%=========================================================================================
%=======================Научная новизна===================================
%=========================================================================================
\begin{frame}{Научная новизна}
\begin{enumerate}
\item Изучены свойства исчисления ПО--формул, влияющие на применимость известных методик повышения эффективности поиска ЛВ.
\item Впервые предложены и реализованы подходы обработки неограниченных переменных, стратегия $k,m$--ограничения, структуры данных экономного представления ПО--формул с дизъюнктивным ветвлением, обработка предиката равенства без прямого представления аксиом равенства в виде ПО--формулы.
\item Впервые, для исчисления ПО--формул успешно применены широко используемые подходы: индексирование термов, параллельные схемы алгоритмов поиска ЛВ, разделение данных, кэширование результатов, системы переписывания термов.
\item Значительно расширен класс решаемых задач при помощи систем АДТ, базирующихся на исчислении ПО--формул. Построены новые варианты решения задач АДТ по сравнению с другими методами АДТ.
\end{enumerate}
\end{frame}


%=========================================================================================
%=======================Научная и практическая значимость===================================
%=========================================================================================
\begin{frame}{Научная и практическая значимость}
\begin{enumerate}
\item Созданы система АДТ и инструментальная среда разработки систем АДТ, направленные на построение ЛВ формул исчисления ПО--формул первого порядка.
\item Выделены классы задач, для которых созданная система является более эффективной, чем современные высокопроизводительные системы АДТ и предложены специальные стратегии, повышающие эффективность поиска ЛВ.
\item Реализована инфраструктура тестирования разработанных алгоритмов и программного обеспечения АДТ на тестовых задачах из библиотеки TPTP.
\end{enumerate}
\end{frame}

%=========================================================================================
%=======================Научная и практическая значимость (продолжение) ===================================
%=========================================================================================
\begin{frame}{Научная и практическая значимость}
Исследования поддержаны Федеральной целевой программой, базовым финансированием СО РАН и грантом РФФИ:
Федеральная целевая программа ``Научные и научно--педагогические кадры инновационной России'' на 2009-2013 годы, госконтракт № П696;
базовые проекты научно--исследовательской работы ИДСТУ СО РАН, Проект IV.31.2.4., № гос. регистрации: 01201001350, программа IV.31.2., Проект IV.32.1.2., № гос. регистрации: 01201001346;
РФФИ, № 08-07-98005-р\_Сибирь\_а;
программа ``Университетский кластер''.

Разработанные программные средства используются в учебном процессе в Институте математики, экономики и информатики Иркутского государственного университета (ИМЭИ ИГУ), Научно-исследовательском Иркутском государственном техническом университете (НИ ИрГТУ).
\end{frame}

%=========================================================================================
%=======================Результаты выносимые на защиту ===================================
%=========================================================================================
\begin{frame}{Результаты выносимые на защиту}
\begin{enumerate}
\item Разработаны новые программные методики и технологии эффективного поиска ЛВ в исчислении ПО--формул: специальные структуры представления ПО--формул, стратегии поддержки неограниченных переменных, $k,m$--огра\-ни\-че\-ние, алгоритмическая поддержка предиката равенства без явного использования аксиом равенства. Разработанные методики реализуют полезные свойства исчислений ПО--формул в программных системах АДТ.

\item Успешно адаптированы и применены существующие методики повышения эффективности поиска ЛВ для исчисления ПО--формул: индексирование термов, параллельные схемы алгоритмов, разделение данных для термов. Адаптация сохраняет полезные свойства исчислений ПО--формул.

\item Реализована новая версия программной системы для эффективного поиска ЛВ в исчислении ПО--формул первого порядка с функциональными символами и предикатом равенства. Создан программный инструментарий и технологии разработки специализированных версий программы АДТ, направленные на решение определенных классов задач АДТ. % В работе слабо представлено. Инстр - в третьей, а конкретная адаптация - четвертая глава.

\item Разработана инфраструктура тестирования алгоритмов и стратегий ЛВ для созданной программной системы АДТ на задачах из библиотеки TPTP. Выделены классы задач, которые система решает эффективнее, чем другие передовые системы АДТ. Значительно расширен класс успешно решаемых задач по сравнению с предыдущими системами АДТ, базирующимися на исчислении ПО--формул.
\end{enumerate}
\end{frame}


%=========================================================================================
%=======================Публикации ===================================
%=========================================================================================
\begin{frame}{Публикации}

\end{frame}


%=========================================================================================
%=======================Апробация ===================================
%=========================================================================================
\begin{frame}{Апробация}

\end{frame}


%=========================================================================================
%=======================Личный вклад автора ===================================
%=========================================================================================
\begin{frame}{Личный вклад автора}
\begin{enumerate}
\item Разработка оригинальных стратегий эффективного поиска ЛВ в исчислении ПО--формул [1,2,5,7,8].
\item Адаптация существующих методик эффективного поиска ЛВ в исчислении ПО--формул [1,3,6,7].
\item Реализация структур данных для представления ПО--формул [1,5,7,8].
\item Реализация программной системы АДТ ПО--формул [2,3,5,7,8].
\item Тестирование разработанной системы АДТ на задачах из библиотеки TPTP [2,5,7].
\end{enumerate}
\end{frame}


%=========================================================================================
%=======================Соответствие паспорту ===================================
%=========================================================================================
\begin{frame}{Соответствие паспорту специальности}

\end{frame}


%=========================================================================================
%=======================Структура ===================================
%=========================================================================================
\begin{frame}{Структура и объем диссертации}
Диссертация состоит из введения, 4 глав, заключения, приложений и списка литературы из 100 наименований. Общий объём работы -- 163 страницы, из которых 131 страница -- основной текст, включающий 13 рисунков и 7 таблиц.
\end{frame}


%=========================================================================================
%=======================Содержание ===================================
%=========================================================================================
\begin{frame}{Содержание работы}

\end{frame}


%=========================================================================================
%=======================Глава 1===================================
%=========================================================================================
\begin{frame}{Глава 1}

\end{frame}

%======================= ===================================
\begin{frame}{}

\end{frame}

%======================= ===================================
\begin{frame}{}

\end{frame}

%======================= ===================================
\begin{frame}{}

\end{frame}

%======================= ===================================
\begin{frame}{}

\end{frame}

%======================= ===================================
\begin{frame}{}

\end{frame}

%======================= ===================================
\begin{frame}{}

\end{frame}

%======================= ===================================
\begin{frame}{}

\end{frame}

%======================= ===================================
\begin{frame}{}

\end{frame}

%======================= ===================================
\begin{frame}{}

\end{frame}

%======================= ===================================
\begin{frame}{}

\end{frame}

%======================= ===================================
\begin{frame}{}

\end{frame}

%======================= ===================================
\begin{frame}{}

\end{frame}

%======================= ===================================
\begin{frame}{}

\end{frame}

%======================= ===================================
\begin{frame}{}

\end{frame}

%======================= ===================================
\begin{frame}{}

\end{frame}

%======================= ===================================
\begin{frame}{}

\end{frame}

%======================= ===================================
\begin{frame}{}

\end{frame}

%======================= ===================================
\begin{frame}{}

\end{frame}

%======================= ===================================
\begin{frame}{}

\end{frame}

%======================= ===================================
\begin{frame}{}

\end{frame}

%======================= ===================================
\begin{frame}{Заключение}

\end{frame}

\end{document}