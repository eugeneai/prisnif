\chapter{Логическое моделирование вариантов принятия решения по рациональному использованию лесных ресурсов}

Содержание данной главы базируется на \cite{dissche} и \cite{dissAsya}, проводится обобщение результатов, представленных в данных диссертациях.


\section*{Введение}
Российская Федерация обладает большими запасами лесных ресурсов (ЛР), которые размещены неравномерно, местами истощены, а значительная часть территории страны относится к особо охраняемым территориям, где ведение хозяйственной деятельности ограничено. Значительная часть сибирских и дальневосточных регионов РФ  являются источником лесных ресурсов, которые составляют основную экспортную составляющую для этих регионов.

Проблема формирования политики использования ЛР является чрезвычайно важной задачей для лиц, принимающих решения (ЛПР) в управлении лесопромышленным регионом. ЛПР  в процессе принятия решения сталкивается с задачами, которые являются «антиинтуитивными». Под «антиинтуитивными» решениями понимаются решения, которые не являются «очевидно хорошими» на взгляд эксперта, т.е. решения, которые требуют специального исследования. Эффективность принимаемых ЛПР решений в первую очередь зависит от объема, вида и качества исходных данных о состоянии ЛР, а также прогнозов развития ЛР в зависимости от принимаемых ЛПР решений (политики заготовки ЛР).

На процесс принятия решения часто воздействуют
различные случайные параметры, усложняющие
процедуру принятия решения. Недостаток информации
об их распределении (сложность их измерения)
приводит к необходимости принятия гипотез
как об области  изменения данных параметров,
так и о характере их распределения (о функции
распределения вероятностей). Правильность
используемых гипотез необходимо проверять
с помощью методов оценки статистических гипотез.
Проблемы принятия решений с недетерминированными
параметрами называют проблемами принятия решений
в условиях недостатка информации. Чем меньше
информации об исследуемом объекте есть, тем
больше может оказаться различие между ожидаемым
и действительным результатами принимаемых
решений в целом [41].

Современные системы поддержки принятия решения
(СППР), возникшие как естественное развитие
и продолжение управленческих информационных
систем и систем управления базами данных, представляют
собой системы, максимально приспособленные
к решению задач повседневной управленческой
деятельности, и являются инструментом, призванным
оказать помощь ЛПР. С помощью СППР могут решаться
неструктурированные и слабоструктурированные
многокритериальные задачи.

СППР, как правило, являются результатом мультидисциплинарного
исследования, включающего теории баз данных,
искусственного интеллекта [53, 72], интерактивных
компьютерных систем [73], методов имитационного
моделирования [54, 76].

Ранние определения СППР (в начале 70-х годов
ХХ века) отражали следующие три момента:
\begin{enumerate}
\item возможность оперировать с неструктурированными
или слабоструктурированными задачами, в отличие
от задач, с которыми имеет дело исследование
операций;
\item интерактивные автоматизированные, т.е. реализованные
на базе компьютера, системы;
\item разделение данных и моделей.
\end{enumerate}

В данной главе под СППР понимется интерактивная автоматизированная система, которая помогает ЛПР использовать имеющиеся данные и математические модели для идентификации и решения поставленных перед ним задач и принятия управленческих и других решений.

СППР являются человеко-машинными программными объектами, которые позволяют ЛПР использовать разнообразные методы (данные, знания, объективные и субъективные модели) для анализа и решения слабоструктурированных и неструктурированных задач. Идея СППР возникла как попытка автоматизации естественных человеческих действий по анализу имеющейся информации, планированию действий и т.п. с целью решения конкретной поставленной задачи.

Основной частью СППР является генератор решений
(рис. 1.1), который действует на основании исходных
данных (блок «Данные») об объекте исследования
и описания постановки задачи. В процессе принятия
решения генератор решений взаимодействует
с подсистемой искусственного интеллекта, которая,
в свою очередь, использует базу знаний для построения
базовой структуры модели объекта исследования
по известным исходным данным [18]. Также в базу
знаний включаются разделы, формализующие некоторые
способы изменения состояния объекта, и эвристики,
используемые математиком-исследователем в
процессе построения модели, что обеспечивает
подстройку модели к специфическим свойствам
объекта. Таким образом, подсистема искусственного
интеллекта используется для идентификации
математической модели объекта и оценивания
вариантов решения [33, 81].

На этапе идентификации модели используются
данные об исследуемом объекте, которые хранятся
в базе данных (БД). В базе знаний хранятся знания
эксперта об особенностях объекта исследования
[3, 10, 28]. Далее  производится идентификация параметров
модели объекта исследования, вычисление начальных
условий, по выбранной модели рассчитываются
прогнозы состояния объекта [7]. При этом могут
просчитываться различные сценарии динамики
объекта, в зависимости от комбинации начальных
параметров.

Рассчитываются критерии анализа сценариев,
над которыми далее проводится многокритериальная
оптимизация (МКО) [27]. МКО требуется для сужения
исходного набора сценариев по заданному набору
критериев [40, 97]. Если результаты расчетов имеют
пространственную привязку, то они отображаются
по запросу пользователя в виде картографического
произведения [43, 65] (рис. 1.2).

\section{}
