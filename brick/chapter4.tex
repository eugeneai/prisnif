\chapter{Применение программной системы}
\label{part:examples}

В данной главе продемонстрированы примеры применения разработанных инструментальных средств  для решения практических задач. В частности, рассматривается пример 

%-----------------------------------------------------------------------------------------
%---------------------Эксперименты, тестирование, сравнение------------------------------
%-----------------------------------------------------------------------------------------
\section{Применимость}
Как видно, каждый узел дерева состояний вывода олицетворяет шаг вывода. Кроме того, узел содержит очень много дополнительной информации, т.е. потребляет ресурсы памяти, причем с каждым шагом вывода потребление может увеличиваться. Таким образом с точки зрения потребляемости ресурсов прувер эффективнее использовать на задачах имеющих крупноблочную структуру, т.е., при формализации конъюнкты действительно должны быть конъюнктами, а не вырожденными одноэлементными или пустыми, древовидная структура действительно должна быть древовидной, глубокой, а не с глубиной 4. Таким образуом получится что каждый узел будет “обслуживать” довольно объемные и информативные куски данных. Одновременно с этим язык ПО--формул как раз и позиционируется как крупноблочный. Отсюда предположительно успех может возникнуть в задачах, формализуемых в языке ПО--формул таким образом, чтоб как можно сильнее выпячивалась крупноблочность. В самой крупной библиотеке формализованных задач из 10000 задач оказалось что 2000 обладают этим свойством. Поэтому тестирование и сравнение будет проводиться в следующих выборках: задачи которые предположительно наиболее удачно подходят для нашего прувера; и задачи которые предположительно являются неподходящими. 

С другой стороны задачи с неограниченными переменными приводят к использования стратегии ленивой конкретизации, что в конечном итоге может усложнить прозрачность логического вывода. Отсюда к перечисленным выше выборкам будут добавлены задачи с ограниченным и неограниченными переменными.



%=============================== ТЕСТИРОВАНИЕ И СРАВНЕНИЕ======================
\section{Тестирование и сравнение}

Тестирование системы проводилось на задачах из библиотеки TPTP (www.tptp.org). Данная библиотка является де-факто стандартом для тестирования систем АДТ. На момент написания работы библиотека содержала свыше 20000 задч, формализованных на языке предикатов первого порядка (FOF: first-order formula), конъюнктивной нормальной формы (CNF: conjunctive normal form), TFF.

Все задачи классифицированы по некоторым параметрам, а именно:
\begin{enumerate}
\item Рейтинг. Измеряется числом от 0.0 до 1.0. Задача с рейтингом 0 может быть решена любым проверенным прувером внесённым в библиотеку. Задача с рейтингом 1.0 ещё не была решена ниодной системой. Внесенные в систему пруверы именуются как state-of-the-art systems. Задачи с рейтингом выше 0.05 уже относятся к классу difficult и считаются весьма сложными. Тестирование имеет смысл проводить по всем видам рейтинга от 0.0 до 1.0.
\item Предметная область задачи. Например, теоремы математического анализа (ANA), алгебры (ALG), геометрии (GEO), головоломки (PUZ), медицины (MED), и др.
\item Количественные характеристики формулы. Например, количество предикатов, функциональных символов, наличие предиката равенства, общий объем формулы.
\item Важным параметром формулы являетя её статус: теорема (theorem), выполнима (satisfiable), невыполнима (unsatisfiable), countersatisfiable, неизвестно (unknow).
\end{enumerate}

Теоремы выводимы в исчислении ПО--формул, выполнимые задачи невыводимы.

%-------------------
Хорошая эффективность продемонстрирована на задачах из области геометрии. 


\subsection{Задача из медицины}
\paragraph{MED002+1.p}
Задача из \cite{med1}. В TPTP она имеет номер MED002+1.p и рейтинг 0.38.
Формулруется следующим образом: <<Whether or not patients with nearly-exhausted production of glucose in the B-cells are cured with a biguanide and sulfonylurea combination therapy>>. Используется следующий набор аксиом ... и предположение ...

\paragraph{MED009+1.p}
After unsuccessful treatment with single oral anti-diabetic for patients with QI greater equal than 27 medical management moves to next step.

\subsection{Задачи геометрии}
Рассмотрим более подробную информацию о задачах из геометри и сравнеии с другими системами. Полный список решенных задач представлен в приложении.

GEO170+1; 8, 0.01c; 0.04

GEO170+3; 862, 0.3с; 0.17

GEO171+1; ;0.08

GEO171+2; 5, 0.002с; 0.04

GEO173+1; 9, 0.006с; 0.08

GEO173+3; 28, 0.01с; 0.12

GEO175+1; 19, 0.01; 0.12

GEO184+1; 9, 0.03; 0.08

GEO184+3; 17, 0.006; 0.21

GEO185+2; 42, 0.007; 0.04

GEO185+3; 6, 0.005; 0.12

GEO186+3; 17, 0.005; 0.17

GEO187+3; 1653, 0.6с; 0.21

GEO188+3; 97, 0.02с; 0.12

GEO189+3; 5, 0.003; 0.12

GEO190+3; 281, 0.07; 0.12

GEO196+2; 188, 0.02; 0.21

GEO196+3; 23, 0.009; 0.17

GEO197+3; 8, 0.006; 0.17

GEO198+3; 26, 0.01; 0.17

GEO199+3; 13, 0.01; 0.17

GEO200+3; 62, 0.01; 0.12

GEO201+3; 13, 0.05; 0.25

GEO202+3; 96, 0.02; 0.29

GEO203+1; 10, 0.007; 0.12

GEO204+2; 6, 0.01; 0.04 

GEO205+3; 12, 0.01; 0.33

GEO206+1; 33, 0.01; 0.08

GEO206+2; 35, 0.005; 0.04

GEO206+3; 37, 0.01; 0.12

GEO207+1; 2, 0.001; 0.08

GEO207+2; 2, 0.001; 0.04

GEO207+3; 19, 0.005; 0.04

GEO208+1; 9, 0.01; 0.08

GEO208+2; 9, 0.01; 0.04

GEO208+3; 81, 0.02; 0.04

GEO209+1; 45, 0.08; 0.08 

GEO209+2; 76, 0.01; 0.08

GEO209+3; 242, 0.06; 0.08

GEO210+1; 19, 0.01; 0.08

GEO210+2; 25, 0.009; 0.17

GEO210+3; 17, 0.007; 0.08

GEO211+1; 6, 0.001; 0.04

GEO211+2; 4, 0.01; 0.04

GEO211+3; 201, 0.01; 0.29 *

GEO212+1; 27, 0.005; 0.04

GEO212+2; 8, 0.01; 0.08

GEO212+3; 5, 0.01; 0.29 *

GEO213+1; 8, 0.01; 0.08

GEO213+2; 8, 0.01; 0.08

GEO213+3; 5, 0.01; 0.29 *

GEO214+1; 8, 0.01; 0

GEO214+2; 16, 0.01; 0.04

GEO214+3; 13, 0.01; 0.29 *

GEO215+1; 22, 0.007; 0.17

GEO215+2; 48, 0.016; 0.25

GEO215+3; 139, 0.03; 0.29 *

GEO216+1; 4, 0.01; 0.08

GEO216+2; 4, 0.01; 0.04

GEO216+3; 94, 0.04; 0.04

GEO217+3; 26, 0.01; 0.08

GEO218+1; 6, 0.01; 0.08

GEO218+2; 6, 0.01; 0.08

GEO218+3; 50, 0.019; 0.12

GEO219+1; 10, 0.01; 0.08

GEO219+2; 12, 0.01; 0.04

GEO219+3; 5, 0.01; 0.08

GEO220+1; 29, 0.01; 0.04

GEO220+2; 18, 0.01; 0.04

GEO220+3; 14, 0.003; 0.08

GEO221+1; 9, 0.01; 0.17

GEO221+3; 5, 0.01; 0.12

GEO222+2; 99, 0.01; 0.25

GEO222+3; 27, 0.01; 0.17;

GEO223+1; 9, 0.01; 0.21

GEO223+3; 18, 0.01; 0.17

GEO225+1; 6, 0.01; 0.08

GEO225+3; 134, 0.05; 0.12

GEO226+1; 6, 0.01; 0.12

GEO226+2; 57, 0.01; 0.12

GEO226+3; 2622, 0.62; 0.08

GEO228+1; 1087, 0.27; 0.04

GEO232+1; 11, 0.01; 0

GEO235+1; 11, 0.01; 0

GEO236+1; 11, 0.01; 0.04

GEO237+1; 18, 0.01; 0

GEO238+1; 97, 0.03; 0.04

GEO239+1; 11, 0.03; 0.04

GEO240+1; 11, 0.03; 0.04

GEO241+1; 12, 0.03; 0.08

GEO242+1; 11, 0.03; 0.12

GEO243+1; 11, 0.03; 0.04

GEO244+1; 11, 0.03; 0.04

GEO245+1; 11, 0.03; 0.04

GEO246+1; 11, 0.03; 0.04

GEO247+1; 11, 0.03; 0.04

GEO248+1; 11, 0.03; 0.04

GEO249+1; 11, 0.03; 0.04

GEO250+1; 12, 0.03; 0.04

GEO251+1; 11, 0.03; 0.04

GEO252+1; 11, 0.03; 0.04

GEO253+1; 17, 0.03с; 0

GEO256+1; 11, 0.03с; 0.04

GEO257+1; 11, 0.03с; 0.04

GEO257+3; 22, 0.03с; 0.08

GEO261+1; 209, 0.05с; 0.08

GEO262+1; 252, 0.05с; 0.04

GEO263+1; 249, 0.05с; 0.08

GEO264+1; 11, 0.05с; 0.04



\paragraph{GEO222+3.p}
A line L is parallel to the line, that is orthogonal to the orthogonal to L through A, and goes through A as well.



\subsection{Задачи верификации}

\subsection{Задачи головоломки}

\subsection{Крупные задачи}
Рассмотрим поведение системы при решении некоторых крупных задач, формализация которых превышает 1мб. Как правило, это задачи с формализацией очень большого количества аксиом, многие из которых не пригождаются в выводе, т.е. ответы на многие вопросы заведомо не приносят никакой пользы, более того, наоборот, захламляют вывод. Основной тактикой при решении таких задач является во-первых повсеместное использование экономии памяти, тщательное использование собранной статистики для удаления ненужных фактов и вопросов, и использование знаний о задаче, для выявленя заведомо плохих ветветй вывода.

\paragraph{KRS...}
Формулировка:

Данная задача содержит N групп аксиом, по M штук. Формализация в языке ПО--формул составляет .. байт. Количество вопросов ... из них с неограниченными переменными ..., глубоких ..., с дизъюнктивным ветвлением ..., целевых ... . Количество атомов ..., количество функциональных символов... 


%------------------------------
\subsection{Разное}
Задачи которые стоит посмотреть:
SYN457+1.p

Из решенных задач время быстрее чем у Otter. Оттер выводит системную инфу всякую, поэтому думаю сопоставимо время на самом деле.

Задача; Рейтинг; Статус (Решил ли прувер); Комментарий.

ALG211+1; 0.12; Теорема (Да); 0.05 и 40 шагов.
MGT001; 0.12; Теорема (Да); 0.03с и 33 шага.
MGT008; 0.12; Теорема (Да); 0.007с и 6 шагов.
MGT010; 0.08; Теорема (Да); 0.007с и 13 шагов.*
MGT015; 0.12; Теорема (Да); 0.1с и 7 шагов.
MGT007; 0.12; Теорема (Да); 0.01с и 25 шагов, есть ветвление.
GRA014; 0.00; Теорема (Да); 1.2с и 5201 шагов. У Otter уходит 7 секунд. При этом вывод системной инфы не такой огромный.
SET913+1; 0.04; ..

\subsection{Прошлые задачи}

Паровокй каток, телескоп, леса, бутаков...

%=========================ВЫВОДЫ======================
\section{Выводы}
Система протестирована на разных классах задач, в основном взятых из библотеки TPTP. Манипуляцию с включением и отключением стратегий показывают что они положительно влияют на эффективность поиска ЛВ. В некоторых случаях, включение параллельного режима вызывает обратный эффект, т.е. замедление, это связано с тем что решаемая задача сама по себе достаточно проста,и  накладные расходы связанные с созданием дополнительных процессов, копирования данных и т.д. превышает полезный ресурс направленный на опровержение баз. 

Сравнение с иными системами АДТ, показывает что разработанная система выигрывает по времени чаще всего при решении таких задач, формализация которых содержит достатончо объёмные конъюнкты, экзистенциальные переменные и глубокие вопросы. 

Значительный прирост эффективности достигается за счёт кэширования результатов. Опыты показывают что во время поиска ЛВ производится множество дублирующихся шагов и добавление повторяемой информации.

Выигры по сравнению с другими системами выражается меньшим временем решения задач средней или лёгкой сложности. Системы мирового уровня имеют приоритет при решении задач крайне высокого рейтинга (в случае библиотки TPTP, выше 0.8).  



%%% Local Variables: 
%%% mode: latex
%%% TeX-master: "dis"
%%% End: 
