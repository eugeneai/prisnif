\chapter{Применение программной системы}
\label{part:examples}

В данной главе продемонстрированы примеры применения разработанных инструментальных средств  для решения практических задач. В частности, рассматривается пример 

%-----------------------------------------------------------------------------------------
%---------------------Эксперименты, тестирование, сравнение------------------------------
%-----------------------------------------------------------------------------------------
\section{Применимость}
Как видно, каждый узел дерева состояний вывода олицетворяет шаг вывода. Кроме того, узел содержит очень много дополнительной информации, т.е. потребляет ресурсы памяти, причем с каждым шагом вывода потребление может увеличиваться. Таким образом с точки зрения потребляемости ресурсов прувер эффективнее использовать на задачах имеющих крупноблочную структуру, т.е., при формализации конъюнкты действительно должны быть конъюнктами а вырожденными одноэлементными или пустыми, древовидная структура действительно должна быть древовидной, глубокой, а не с глубиной 4. Таким образуом получится что каждый узел будет “обслуживать” довольно объемные и информативные куски данных. Одновременно с этим язык ПО-формул как раз и позиционируется как крупноблочный. Отсюда предположительно успех может возникнуть в задачах, формализуемых в языке ПО-формул таким образом, чтоб как можно сильнее выпячивалась крупноблочность. В самой крупной библиотеке формализованных задач из 10000 задач оказалось что 2000 обладают этим свойством. Поэтому тестирование и сравнение будет проводиться в следующих выборках: задачи которые предположительно наиболее удачно подходят для нашего прувера; и задачи которые предположительно являются неподходящими. 

С другой стороны задачи с неограниченными переменными приводят к использования стратегии ленивой конкретизации, что в конечном итоге может усложнить прозрачность логического вывода. Отсюда к перечисленным выше выборкам будут добавлены задачи с ограниченным и неограниченными переменными.

%=============================== ТЕСТИРОВАНИЕ И СРАВНЕНИЕ======================
\section{Тестирование}

Тестирование системы проводилось на задачах из библиотеки TPTP (www.tptp.org). Данная библиотка является де-факто стандартом для тестирования систем АДТ. На момент написания работы библиотека содержала свыше 20000 задч, формализованных на языке предикатов первого порядка (FOF: first-order formula), конъюнктивной нормальной формы (CNF: conjunctive normal form), TFF.

Все задачи классифицированы по некоторым параметрам, а именно:
\begin{enumerate}
\item Рейтинг. Измеряется числом от 0.0 до 1.0. Задача с рейтингом 0 может быть решена любым прувером внесённым в библиотеку. Задача с рейтингом 1.0 ещё не была решена ниодной системой. Задачи с рейтингом выше 0.5 считаются весьма сложными. Тестирование имеет смысл проводить по всем видам рейтинга от 0.0 до 1.0.
\item Предметная область задачи. Например, теоремы математического анализа (ANA), алгебры (ALG), геометрии (GEO), головоломки (PUZ), медицины (MED), и др.
\item Количественные характеристики формулы. Например, количество предикатов, функциональных символов, наличие предиката равенства, общий объем формулы.
\item Важным параметром формулы являетя её статус: теорема (theorem), выполнима (satisfiable), невыполнима (unsatisfiable), countersatisfiable, неизвестно (unknow).
\end{enumerate}

Теоремы выводимы в исчислении ПО--формул, выполнимые задачи невыводимы.


\section{Конкретные задачи}

\subsection{Задачи без неограниченных переменных}
Из решенных задач время быстрее чем у Otter. Оттер выводит системную инфу всякую, поэтому думаю сопоставимо время на самом деле.

Задача; Рейтинг; Статус (Решил ли прувер); Комментарий.

ALG211+1; 0.12; Теорема (Да); 0.05 и 40 шагов.

MGT001; 0.12; Теорема (Да); 0.03с и 33 шага.

MGT008; 0.12; Теорема (Да); 0.007с и 6 шагов.

MGT010; 0.08; Теорема (Да); 0.007с и 13 шагов.*

MGT015; 0.12; Теорема (Да); 0.1с и 7 шагов.

MGT007; 0.12; Теорема (Да); 0.01с и 25 шагов, есть ветвление.

GRA014; 0.00; Теорема (Да); 1.2с и 5201 шагов. У Otter уходит 7 секунд. При этом вывод системной инфы не такой огромный.

GEO175+1; 0.12; Теорема (Да); 0.01 и 19 шагов.

GEO175+2; 0.12; Теорема (Да);

GEO177+1; 0.25; Теорема (Да); 0.01 и 41 шаг.

GEO178+1; 0.04; Теорема (Да); 0.01 и 42 шага.

GEO179+1; 0.08; как обычно

GEO180+1; 0.21;

GEO180+2; 0.17;

GEO183+1; 0.08; ; 28 шагов.

GEO184+1; 0.08; ; 9 шагов.

GEO186+1; 0.12;

GEO187+1; 0.21;

GEO188+1; 0.17;

GEO189+1; 0.25;

GEO190+1; 0.21;

GEO195+1; 0.29;

194+1, 193+1, 197+1, 198+1, 200+1, 201+1, 202+1, 203+1, 204-216, 236,   

SET913+1; 0.04; ..







%%% Local Variables: 
%%% mode: latex
%%% TeX-master: "dis"
%%% End: 
