\chapter*{Введение}

%===============================================================================
%===============================================================================
%============================================================================================
%-------------------------------------АКТУАЛЬНОСТЬ--------------------------------------------
%добавить ссылки
\section*{Актуальность исследования}
%Развитие аппарата математической логики и вычислительных систем естественным образом привело в середине XX века к созданию первых систем автоматизации поиска логического вывода (ЛВ) \cite{Newell1, Newell2, WangHao}, также называемого автоматическим доказательством теорем (АДТ). В данном контексте, как теоремы, так и ЛВ понимаются как формальные математические объекты. Теорема теории $T$ --- это формула теории $T$, для которой существует логический вывод в теории $T$ \cite{mendelson}. Логический вывод в теории $T$ --- последовательность формул, состоящая либо из аксиом $T$, либо из формул $T$ полученных из предшествующих по правилам вывода заданным в $T$ \cite{mendelson}. Множество допустимых формул (формальный язык), аксиомы и правила вывода, образуют вместе формальную систему (исчисление). Таким образом, в общем случае, назначение систем АДТ --- это автоматизированное решение любых задач заданного исчисления. Отсюда, системы АДТ представляют собой инструментарий для автоматизированного  решения довольно широких классов задач. Далее, в работе, под ``поиском ЛВ'', как правило, будем иметь ввиду автоматизированный поиск ЛВ.

Одним из подходов к интеллектуализации программных систем является разработка алгоритмов обработки информации, основанных на моделировании процесса рассуждений. Наиболее формализованные подходы базируются на автоматическом построении логического вывода (ЛВ) в некоторой системе формализованных знаний (логического описания предметной области). Программные системы для поиска ЛВ называют системами автоматического доказательства теорем (АДТ), поскольку утверждения, для которых существует ЛВ, являются теоремами (в заданном исчислении).

Формализации некоторых предметных областей, например, аксиоматизации математических теорий, верификация программного и аппаратного обеспечения и др., являются весьма громоздкими, и использование систем АДТ для построения ЛВ становится необходимым. Кроме того, системы АДТ необходимы, например, в системах планирования действий, системах поддержки принятия решений, экспертных системах, где важным фактором явлется время решения задачи и другие критерии. Далее, в работе, под ``поиском ЛВ'', как правило, будем иметь ввиду автоматизированный поиск ЛВ.

В рамках исследования формальных систем были получены следующие результаты:
\begin{enumerate}
\item Разрешимость логики высказываний \cite{mendelson}. Для любого утверждения логики высказываний существует алгоритм проверки его истинности.
\item Полуразрешимость исчислений первого порядка \cite{turing}. Если формула является теоремой, то существуют алгоритм, устанавливающий данный факт. Но если формула не является теоремой, то такого алгоритма, в общем случае, не существует.
\item Полнота исчислений первого порядка \cite{Godel1}. То есть соответствие между семантикой и синтаксисом, позволяющее абстрагироваться от семантики решаемой задачи.
\item Теорема Гёделя о неполноте непротиворечивых формальных систем, которые включают арифметику \cite{Godel2}. Принципиальная невозможность как доказательства, так и опровержения некоторых утверждений.
\end{enumerate}

Хотя логика высказываний является полной и разрешимой, её язык недостаточно выразителен для решения многих проблем. С другой стороны, логики высших порядков являются весьма выразительными, но, тем не менее, часто плохо пригодными для автоматизации поиска ЛВ, поскольку в силу результатов Гёделя неэффективны и неполны. В данном случае, компромиссом являются исчисления первого порядка \cite{Frege, Sourcebook, mendelson}, достаточно выразительные для формализации многих задач, корректные, полные и полуразрешимые. Именно поэтому в основу большинства систем АДТ легли формальные системы первого порядка.

Системы АДТ появились в середине XX века, как результат развития аппарата математической логики и вычислительных систем, и успешно применяются для решения практических задач. Имеются примеры решения открытых математических проблем \cite{McCuneRob}. Более подробно об использовании систем АДТ и о современных лидерах в данной области рассказано в Главе~\ref{basis}

Отметим, что важен не только сам факт установления выводимости формулы и ширина класса решаемых задач, но и затраченные вычислительные ресурсы: время решения задачи, количество шагов вывода, расход оперативной памяти. Для анализа и интерпретации полученного ЛВ важными факторами являются количество шагов вывода и способ решения, позволяющие, в том числе, получить альтернативные варианты решений задач, другие способы интерпретации результатов ЛВ. Наращивание эффективности поиска ЛВ повышает вероятность решения новых открытых проблем, даёт возможность использовать системы АДТ в новых предметных областях. В работах С.Н. Васильева \cite{ICDS2000} развиты теоретические методы применения АДТ в управлении динамическими системами. Масштабное тестирование систем АДТ \cite{tptp} показывает, что далеко не все классы задач покрыты эффективными решателями. Таким образом, проблема повышения эффективности поиска ЛВ и расширения классов решаемых задач системами АДТ является актуальной задачей. Более того, в силу полуразрешимости исчислений первого порядка, данная проблема всегда является актуальной.

%Со временем планка сложности задач повышается. Полный перебор пространства поиска решений крайне неэффективен. Всегда найдутся задачи со сколь угодно большим минимальным ЛВ, существует большое количество ещё в принципе нерешенных проблем \cite{tptp}. Более того, на практике, как правило, важен не только сам факт установления выводимости формулы, но и затраченные вычислительные ресурсы, например, в системах принятия решений или реального времени. А для анализа и интерпретации полученного ЛВ важными факторами является количество шагов вывода и способ решения, позволяющие, в том числе получить альтернативные взгляды на решенную задачу, иные способы интерпретации результатов ЛВ. Наращивание эффективности поиска ЛВ повышает вероятность решения новых открытых проблем, даёт возможность использовать системы АДТ в новых предметных областях, в частности в работах С.Н. Васильева \cite{ICDS2000} развиваются теоретические методы применения АДТ в управлении динамическими системами. Масштабное тестирование систем АДТ \cite{tptp} показывает, что далеко не все классы задач покрыты эффективными решателями. Таким образом, проблема повышения эффективности поиска ЛВ и расширения классов решаемых задач системами АД является актуальной задачей. Более того, в силу полуразрешимости исчислений первого порядка, данная проблема всегда является актуальной

Для оценки систем АДТ используются следующие формальные критерии эффективности поиска ЛВ:
\label{pg:criterion}
\begin{enumerate}
\item{Время решения задачи.} Какое время затрачено системой АДТ для поиска ЛВ, либо установления факта невыводимости формулы.
\item{Количество шагов вывода.} Количество шагов, из которых состоит найденный ЛВ (количество формул в цепочке ЛВ).
\item{Расход памяти.} Сколько памяти компьютера затрачено системой АДТ в процессе решения задачи.
\item{Ширина класса решаемых задач.} Объединение класса новых задач решенных системой и класса задач, для которых предложен новый способ решения.
\end{enumerate}

Очевидно, что повышение эффективности поиска ЛВ заключается в уменьшении численных характеристик критериев 1--3, и увеличении 4. Далее, в работе, говоря об эффективности, будем иметь ввиду один или несколько перечисленных критериев.

%Таким образом, с учетом широкой применимости систем АДТ и потребности в эффективных системах АДТ, проблема повышения эффективности поиска ЛВ системами АДТ является актуальной задачей. Более того, в силу полуразрешимости исчислений первого порядка, данная проблема всегда является актуальной.

В качестве теоретического базиса разработанной, в данной диссертационной работе, системы АДТ выбрано исчисление позитивно--образованных формул (ПО--формул) \cite{ICDS2000, Vas1995}, разработанное под руководством академика Васильева~С.Н., к.ф-м.н. Жерлова~А.К. на основе языка типово--кванторных формул, изначально использовавшегося для формализации свойств динамических систем и синтеза теорем в методе векторных функций Ляпунова. Опишем свойства ПО--формализма \cite{ICDS2000}, в контексте АДТ и повышения эффективности поиска ЛВ, повлиявшие на выбор именного этого исчисления в качестве теоретического базиса разработанной системы:
\begin{enumerate}
\item {Машинно--ориентированность} обусловлена тем, что ПО--формулы достаточно однородны и \emph{хорошо структурированы}, а исчисление ПО--формул включает \emph{единственное} правило вывода и \emph{единственную} схему аксиом.
\item {Сокращение шагов ЛВ} возможно благодаря крупноблочной структуре формулы и крупноблочности правила вывода. Кроме того, ПО--формулы лучше сохраняют исходную эвристическую структуру знания о задаче при переводе формализации задачи из исчисления предикатов первого порядка, чем, например, конъюнктивная нормальная форма, это позволяет использовать дополнительные структурные эвристики в процессе поиска ЛВ.
\item {Сокращение расхода памяти} возможно за счёт компактного представления ПО--формул и использования $\exists$--переменных в явном виде. Но эффективность данного критерия повышается в основном на уровне реализации системы АДТ.
\item {Сокращение времени поиска ЛВ} не заложено в само исчисление, но решается на уровне реализации системы АДТ. Тем не менее, сокращение числа шагов вывода оказывает положительное влияние на качество результата.
\item {Расширение класса решаемых задач} обусловлено тем, что ЛВ в исчислении ПО--формул отличается от ЛВ других исчислений, а значит возможно получение новых способов решения задач. Кроме того, выразительность языка ПО--формул позволяет использовать его более широко, в смысле предметных областей, относительно других методов АДТ.
\end{enumerate}

Разработанные раннее системы АДТ, базирующиеся на исчислении ПО--формул, либо являются неэффективным, либо носят чисто демонстрационный характер \cite{dissChe}, либо направленны на решение очень узкого класса задач. Более подробно об исчислении ПО--формул и системах, разработанных на их основе рассказано в Главе~\ref{basis} Таким образом, проблема повышения эффективности поиска ЛВ актуальна и для систем АДТ, базирующихся на исчислении ПО--формул.

%Для повышения эффективности поиска ЛВ нами используются следующие подходы:

%Исчисление ПО--формул, даёт фундаментальное сокращение шагов вывода, в силу крупноблочности правила вывода и иерархической структуры; не основано на методе резолюции, т.е. даёт иной подход к решению задач; сохраняет эвристическую структуру знания о задаче, а значит более приспособлено к усвоению эвристик, чем метод резолюции.

%Исходя из изложенных проблем общего характера и классификации существующих систем АДТ для исчисления ПО--формул, в данной работе акцентируется внимание на разработке системы относящейся к 1 и 3 классу, т.е. с одной стороны повышение производительности автоматической части системы, и разработка инфраструктуры для взаимодействия с пользователем и подключением дополнительных стратегий решения, направленных на конкретную задачу.


%Многие современные системы АДТ разрабатываются уже много лет (есть примеры 30-ти летнего опыта), в них внедрено огромное количество методик. Тем не менее эти методики как правило не носят интеллектный характер, а направлены лишь на эффективную обработку данных (например, индексирование), сокращение потребляемой памяти и т.д. Либо предложен некоторый ограниченный ряд стратегий общего характера.

%Кроме того в силу интеллектности и свойства интерпретируемости вывода в исчислении ПО--формул, было бы интересно разработать методики преобразования формального вывода к виду, пригодному для понимания человеку.



%===============================================================================
%=================================ОБЪЕКТ===============================
%===============================================================================
% Объект — это процесс или явление, порождающее проблемную ситуацию и взятое исследователем для изучения. Предмет — это то, что находится в рамках, в границах объекта. Объект — это та часть научного знания, с которой исследователь имеет дело. Предмет исследования — это тот аспект проблемы, исследуя который, мы познаем целостный объект, выделяя его главные, наиболее существенные признаки. Предмет диссертационного исследования чаще всего совпадает с определением его темы или очень близок к нему. Объект и предмет исследования как научные категории соотносятся как общее и частное.
\section*{Объект исследования}
Исчисление позитивно--образованных формул и логический вывод в данном исчислении.


%===============================================================================
%=========================ПРЕДМЕТ======================================
%=========================================================================
\section*{Предмет исследования}
Свойства языка и исчисления ПО--формул, современных структур данных и алгоритмов поиска ЛВ в исчислениях первого порядка, разработка новых структур данных и алгоритмов, обеспечивающих создание программных систем и технологий эффективного поиска ЛВ в исчислениях ПО--формул. Эффективность поиска ЛВ задается следующими критериями: время решения задачи; количество шагов ЛВ; объём использованной оперативной памяти; ширина класса успешно решаемых задач.




%===============================================================================
%==================================ЦЕЛЬ====================================
%===============================================================================
\section*{Цель диссертационной работы}
Цель работы --- разработка программной системы для эффективного поиска логического вывода в исчислении позитивно--образованных формул.


%===============================================================================
%================================ЗАДАЧИ===================================
%===============================================================================
\section*{Основные задачи диссертационной работы}
Для достижения описанной выше цели решаются следующие задачи:
\begin{enumerate}
%\item Анализ свойств языка и исчисления ПО--формул, разработка подхода к адаптации существующих методик построения высокопроизводительных ЛВ для этого исчисления с эффективным использованием оперативной памяти вычислительной системы.
\item Разработка эффективных структур данных представления ПО--формул в памяти компьютера.
\item Создание оригинальных методик (стратегий) эффективного поиска ЛВ в исчислении ПО--формул.
\item Адаптация существующих методик АДТ для исчисления ПО--формул с целью повышения эффективности поиска ЛВ.
\item Разработка эффективной программной системы АДТ, создание инструментальных средств для программирования систем АДТ в исчислении ПО--формул.
\item Апробация разработанных программных средств на решении тестовых и практических задач. Сравнение с другими системами АДТ.
\end{enumerate}


%===============================================================================
%==============================МЕТОДИКА=====================================
%===============================================================================
\section*{Методика исследования}
Методика исследования состоит в применении современных методов математической логики, теории программирования, информационного моделирования, методов искусственного интеллекта.


%===============================================================================
%============================НОВИЗНА=====================================
%=========================================================================
\textbf{Научная новизна работы}
Предложена методика управления поиском логического вывода в исчислении ПО--формул, базирующаяся на обобщении стратегии $k$--опровержения ($k,m$--ограничение), специального варианта стратегии и ограничений выбора элементов Эрбранова универсума (последовательная конкретизация), а также использование структур данных, ориентированных на совместное использование общих участков оперативной памяти (data sharing) базовыми подформулами. Впервые, для исчисления ПО--формул успешно применены широко используемые подходы: индексирование термов, параллельные схемы алгоритмов поиска ЛВ, разделение данных термов, кэширование результатов поиска подстановок. Значительно расширен класс решаемых задач при помощи систем АДТ, базирующихся на исчислении ПО--формул. Построены новые варианты решения задач АДТ по сравнению с другими методами АДТ.

%=======================
\textbf{Практическая значимость полученных результатов}
Созданы система АДТ и инструментальная среда разработки систем АДТ, направленные на построение ЛВ формул исчисления ПО--формул первого порядка. Выделены классы задач, для которых созданная система является более эффективной, чем современные высокопроизводительные системы АДТ. Реализована инфраструктура тестирования разработанных алгоритмов и программного обеспечения АДТ на тестовых задачах из библиотеки TPTP.
%Множество задач решено эффективнее по сравнению с другими системами АДТ. %%% Выделены классы задач, на которых производительность разработанной системы АДТ превышает производительность систем--аналогов.

%Кроме того, немаловажным результатом является возможность использования системы АДТ базирующейся на раннее не использованном исчислении ПО--формул (новый взгляд на задачи).

Исследования, проведенные в рамках  диссертационной работы, выполнены в ФГБОУ ВПО ``Иркутский государственный университет'' и в Институте динамики систем и теории управления СО РАН. Исследования поддержаны Федеральной целевой программой, базовым финансированием СО РАН и грантом РФФИ: Федеральная целевая программа ``Нау\-чные и нау\-чно--пе\-да\-го\-ги\-чес\-кие кадры инновационной России'' на 2009-2013 годы, госконтракт № П696; базовые проекты нау\-чно--иссле\-до\-ва\-тель\-ской работы ИДСТУ СО РАН, Проект IV.31.2.4., № гос. регистрации: 01201001350, программа IV.31.2., Проект IV.32.1.2., № гос. регистрации: 01201001346; РФФИ, № 08-07-98005-р\_Сибирь\_а; программа ``Университетский кластер''.

Разработанные программные средства используются в учебном процессе в Институте математики, экономики и информатики Иркутского государственного университета (ИМЭИ ИГУ), Национальном исследовательском Иркутском государственном техническом университете (НИ ИрГТУ) и в научной деятельности Института динамики систем и теории управления СО РАН (ИДСТУ СО РАН). %Разработан новый вариант курса ``Технологии разработки программного обеспечения'' на кафедре информационных технологий ИМЭИ ИГУ.



%===============================================================================
%===============================================================================
%===============================================================================
\section*{Результаты, выносимые на защиту}
\begin{enumerate}
\item Разработаны новые программные методики и технологии эффективного поиска ЛВ в исчислении ПО--формул: специальные структуры представления ПО--формул, стратегии поддержки неограниченных переменных, $k,m$--огра\-ни\-че\-ние, алгоритмическая поддержка предиката равенства без явного использования аксиом равенства. Разработанные методики реализуют полезные свойства исчислений ПО--формул в программных системах АДТ.

\item Успешно адаптированы и применены существующие методики повышения эффективности поиска ЛВ для исчисления ПО--формул: индексирование термов, параллельные схемы алгоритмов, разделение данных для термов. Адаптация сохраняет полезные свойства исчислений ПО--формул.

\item Реализована новая версия программной системы для эффективного поиска ЛВ в исчислении ПО--формул первого порядка с функциональными символами и предикатом равенства. Создан программный инструментарий и технологии разработки специализированных версий программы АДТ, направленные на решение определенных классов задач АДТ. % В работе слабо представлено. Инстр - в третьей, а конкретная адаптация - четвертая глава.

\item Разработана инфраструктура тестирования алгоритмов и стратегий ЛВ для созданной программной системы АДТ на задачах из библиотеки TPTP. Выделены классы задач, которые система решает эффективнее, чем другие передовые системы АДТ. Значительно расширен класс успешно решаемых задач по сравнению с предыдущими системами АДТ, базирующимися на исчислении ПО--формул.
\end{enumerate}



%===============================================================================
%===============================================================================
%===============================================================================
\section*{Представление работы}
Материалы работы докладывались на:
%\begin{itemize}
Международной конференции ``Мальцевские чтения'', г.Новосибирск, 24-28 августа 2009г.;
Семинаре ИДСТУ СО РАН ``Ляпуновские чтения'', ИДСТУ СО РАН, г.Иркутск, 21-23 декабря 2009г.;
Всероссийской конференции молодых ученых ``Математическое моделирование и информационные технологии'', г.Иркутск, 15-21 марта 2010г.;
Международной конференции ``Облачные вычисления. Образование. Исследования. Разработки'', г.Москва 15-16 апреля 2010г.;
Международном симпозиуме по компьютерным наукам в России. Семинар ``Семантика, спецификация и верификация программ: теория и приложения'', г.Казань, 14-15 июня 2010г.;
4-ой Всероссийской конференции ``Винеровские чтения'', г.Иркутск, 9-14 марта 2011г.
34-ом международном симпозиуме ``MIPRO'', г.Опатия, Хорватия, 23-27 мая 2011г.
4-ой Всероссийской мультиконференции по проблемам управления, с. Див\-но\-мор\-ское, 3-8 октября 2011г.
13-ой национальной конференции по искусственному интеллекту с международным участием (КИИ-2012), г.Белгород, 16-20 октября 2012г.
%\end{itemize}



%===============================================================================
%===============================================================================
%===============================================================================
\section*{Публикации по теме диссертации}
По теме диссертации опубликовано 14 печатных работ, в том числе 3 статьи в журналах, входящих в перечень изданий , рекомендуемых ВАК РФ \cite{mais, distvirt, burvest}, 2 статьи в сборниках научных трудов, 9 работ в сборниках трудов конференций. %5 работ в других сборниках (в том числе из Web of Science), по списку литературы \cite{accs, viner, QUANT4, mipro, semver} и 6 работ в тезисах конференций.

%По теме диссертации опубликовано 3 работы в журналах из перечня рецензируемых научных журналов и изданий ВАК (25.05.2012), в которых публикуются научные результаты диссертации на соискание ученой степени доктора и кандидата наук.
%\begin{enumerate}
%mais
%\item Давыдов А.В., Ларионов А.А., Черкашин Е.А. Об исчислении позитивно-образованных формул для автоматического доказательства теорем // Моделирование и анализ информационных систем. 2010.T. 17, N 4, С. 60--69.
%burvest
%\item Ларионов А.А., Черкашин Е.А., Терехин И.Н. Системные предикаты для управления логическим выводом в системе автоматического доказательства теорем для исчисления позитивно--образованных формул // Вестник Бурятского государственного университета, 2011, выпуск 9. Серия Математика. Информатика. с. 94-98.
%distvirt
%\item Ларионов А.А., Черкашин Е.А. Параллельные схемы алгоритмов автоматического доказательства теорем в исчислении позитивно-образованных формул // Дистанционное и виртуальное обучение. Февраль 2012. No. 2, С. 93-100.
%accs
%\item Davydov A.V., Larionov A.A., Cherkashin E.A. On the calculus of positively constructed formulas for automated theorem proving // Automatic Control and Computer Sciences (AC\&CS). 2011. Volume 45, Issue 7, pp.~402-407.
%\end{enumerate}




%===============================================================================
%===============================================================================
%===============================================================================
%добавить ссылки на литературу
%viner - труды винеровских чтений
%QUANT4 - сборник имэи
%mipro
%mais, burvest, distvirt, accs
\section*{Личный вклад автора}
Автором диссертации лично осуществлены
\begin{enumerate}
\item Разработка оригинальных стратегий эффективного поиска ЛВ в исчислении ПО--формул \cite{mais, burvest, viner, QUANT4, mipro}.
\item Адаптация существующих методик эффективного поиска ЛВ в исчислении ПО--формул \cite{mais, distvirt, viner, mipro}.
%\item Реализация структур данных для представления ПО--формул \cite{mais, viner, mipro, QUANT4}.
\item Реализация программной системы АДТ ПО--формул \cite{distvirt, burvest, viner, mipro, QUANT4}.
\item Тестирование разработанной системы АДТ на задачах из библиотеки TPTP \cite{distvirt, viner, mipro}.
\end{enumerate}

Под руководством к.т.н. Черкашина Е.А. проведена разработка структур данных для представления ПО--формул с использованием разделения оперативной памяти. Совместно с Давыдовым А.В. созданы методики поиска ЛВ с использованием неограниченных переменных.

Из печатных работ, опубликованных диссертантом в соавторстве, в текст глав 2, 3 и 4 диссертации вошли те результаты, непосредственно определяющие творческий вклад автора диссертации на этапах проектирования и разработки программного обеспечения. В перечисленных публикациях все результаты по реализации и использованию программной системы, принадлежат автору.



%===============================================================================
%===============================================================================
%=========================================================================
%-------------------------------------
%05.13.17 Теоретические основы информатики
%Формула специальности:
%Теоретические основы информатики – специальность, включающая исследования процессов создания, накопления и обработки информации; исследования методов преобразования информации в данные и знания; создание и исследование информационных моделей, моделей данных и знаний, методов работы со знаниями, методов машинного обучения и обнаружения новых знаний; исследования принципов создания и функционирования аппаратных и программных средств автоматизации указанных процессов.

%Научное и народнохозяйственное значение решения проблем указанной специальности состоит в создании научных основ современных информационных технологий на базе использования средств вычислительной техники и в ускорении на этой основе научно-технического прогресса.

%Области исследований:

%1. Исследование, в том числе с помощью средств вычислительной техники, информационных процессов, информационных потребностей коллективных и индивидуальных пользователей.
%2. Исследование информационных структур, разработка и анализ моделей информационных процессов и структур.
%3. Исследование методов и разработка средств кодирования информации в виде данных. Принципы создания языков описания данных, языков манипулирования данными, языков запросов. Разработка и исследование моделей данных и новых принципов их проектирования.
%4. Исследование и разработка средств представления знаний. Принципы создания языков представления знаний, в том числе для плохо структурированных предметных областей и слабоструктурированных задач; разработка интегрированных средств представления знаний, средств представления знаний, отражающих динамику процессов, концептуальных и семиотических моделей предметных областей.
%5. Разработка и исследование моделей и алгоритмов анализа данных, обнаружения закономерностей в данных и их извлечениях разработка и исследование методов и алгоритмов анализа текста, устной речи и изображений.
%6. Разработка методов, языков и моделей человекомашинного общения; разработка методов и моделей распознавания, понимания и синтеза речи, принципов и методов извлечения данных из текстов на естественном языке.
%7. Разработка методов распознавания образов, фильтрации, распознавания и синтеза изображений, решающих правил. Моделирование формирования эмпирического знания.
%8. Исследование и когнитивное моделирование интеллекта, включая моделирование поведения, моделирование рассуждений различных типов, моделирование образного мышления.
%9. Разработка новых интернет-технологий, включая средства поиска, анализа и фильтрации информации, средства приобретения знаний и создания онтологии, средства интеллектуализации бизнес-процессов.
%10. Разработка основ математической теории языков и грамматик, теории конечных автоматов и теории графов.
%11. Разработка методов обеспечения высоконадежной обработки информации и обеспечения помехоустойчивости информационных коммуникаций для целей передачи, хранения и защиты информации; разработка основ теории надежности и безопасности использования информационных технологий.
%12. Разработка математических, логических, семиотических и лингвистических моделей и методов взаимодействия информационных процессов, в том числе на базе специализированных вычислительных систем.
%13. Применение бионических принципов, методов и моделей в информационных технологиях.
%14. Разработка теоретических основ создания программных систем для новых информационных технологий.
%15. Исследования и разработка требований к программно-техническим средствам современных телекоммуникационных систем на базе вычислительной техники.
%16. Общие принципы организации телекоммуникационных систем и оценки их эффективности. Разработка научных принципов организации информационных служб по отраслям народного хозяйства. Изучение социально-экономических аспектов информатизации и компьютеризации общества.
%Примечание:

%Специальность не включает исследования в областях: -математическая логика, алгебра и теория чисел; -вычислительная математика; -теория связи, системы и устройства передачи информации по системам связи; -сети, узлы связи и распределения информации; -математическое и программное обеспечение вычислительных машин, комплексов и компьютерных сетей; -вычислительные машины, комплексы, системы и сети; -автоматизированные информационные системы.


\section*{Соответствие паспорту специальности}
Материал диссертации соответствует формуле специальности 05.13.17 -- Теоретические основы информатики. Диссертация посвящена исследованию процессов создания, накопления и обработки информации, представленной в виде фактов и знаний; созданию и исследованию новых моделей знаний, методов представления и обработки знаний; исследованию принципов создания и функционирования программных средств автоматизации представления обработки знаний.

В диссертации получены результаты по следующим пунктам ``Области исследований'': 2. Исследование информационных структур, разработка и анализ моделей информационных процессов и структур; 8. Исследование и когнитивное моделирование интеллекта, включая моделирование поведения, моделирование рассуждений различных типов, образного мышления. Результаты диссертации имеют научное и народнохозяйственное значение для решения проблем повышения производительности обработки знаний в ПО--исчислении, они позволяют развить научные основы современных информационных технологий обработки формализованных знаний средствами вычислительной техники и в ускорении на этой основе научно-тех\-ни\-чес\-ко\-го прогресса.


%Работа относится к области \emph{создания программных средств повышения эффективности} поиска логического вывода. С точки зрения математики, ЛВ понимается как последовательность преобразований формул по заданным правилам вывода. Вычисления, проводимые в ходе поиска ЛВ, являются \emph{символьными} (аналитический подход установления факта выводимости формулы). В программные средства повышения поиска эффективности ЛВ включены \emph{параллельные схемы алгоритмов}. Таким образом, материал диссертации соответствует как формуле специальности 05.13.11 -- Математическое и программное обеспечение вычислительных машин, комплексов и компьютерных сетей, так и следующим пунктам ``Области исследований'': 5. Программные системы символьных вычислений; 8. Модели и методы создания программ и программных систем для параллельной и распределенной обработки данных.



%===============================================================================
%===============================================================================
%=========================================================================
\section*{Структура и объем диссертации}
Диссертация состоит из введения, четырех глав, заключения, списка литературы, приложений. Основной текст изложен на \pageref{pg:main} страницах машинописного текста, полный объем диссертации 164 страницы. В работе содержится 14 рисунков. Список литературы содержит 95 наименований.

В главе 1. дан обзор исторических предпосылок к АДТ, методов и систем АДТ, области их применения. Приведено описание исчисления ПО--формул, являющегося теоретическим базисом разработанной системы АДТ. Выделены основные особенности данного исчисления и проблемы, требующие разрешения.

Глава 2. посвящена представлению методик повышения эффективности поиска ЛВ, разработанным стратегиям поиска ЛВ, базису программной системы. Представлен ряд стратегий, адаптированных из существующих систем АДТ и новые стратегии для исчисления ПО--формул, рассмотрен вопрос решения задач с предикатом равенства.

Глава 3. посвящена аспектам реализации системы. Представлена архитектура системы, описание подсистем, менеджер памяти, системные предикаты управления ЛВ, трансляторы с различных языков представления первопорядковых формул.

В главе 4. представлены методы и результаты тестирования системы. В частности, дано описание библиотеки TPTP, сводная таблица решенных задач из библиотеки TPTP. Приведены примеры, решенных задач и сравнение с передовыми системами АДТ. Даны комментарии по разработанным стратегиям и сделаны выводы о разработанной системе АДТ. %Решена задача о лесах с применением внешних эвристик.

В заключении приводится список результатов, полученных в диссертации. % выносимых на защиту.

Приложение содержит, таблицы решенных задач из библиотеки TPTP, пример формализации задачи в языке ПО--формул, пример протокола вывода для решения одной из задач, акты о внедрении результатов данной диссертационной работы в учебные процессы ИМИЭ ИГУ, НИ ИрГТУ и в научную деятельность ИДСТУ СО РАН.



%===============================================================================
%===============================================================================
%=========================================================================
%--------благодарности-------------
\section*{Благодарности}
Автор благодарит к.т.н. Черкашина~Е.А. за руководство диссертационной работой и помощь в подготовке рукописи, Давыдова~А.В. за ценные указания в работе.


%%% Local Variables:
%%% mode: latex
%%% TeX-master: "dis"
%%% End:
