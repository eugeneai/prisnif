\begin{thebibliography}{99}

%\bibliography{intelle}
%\bibliographystyle{unsrtre}

\bibitem{Bratko} Братко И. Алгоритмы искусственного интеллекта на языке PROLOG / Издательство Вильямс. 2004. 640с.

\bibitem{Butakov1} Бутаков М.И., Курганский В.И. О решении задачи трансляции планированием вычислений на основе позитивно--образованных формул // Системы управления и информационные технологии. 2007. c.~120-123.

\bibitem{ICDS2000} Васильев С.Н. Интеллектное управление динамическими системами. / Васильев С.Н., Жерлов А.К., Федунов Е.А., Федосов Б.Е.  M.:Физматлит, 2000.

\bibitem{Vas1995} Васильев С.Н. Об исчислениях типово-кванторных формул / Васильев С.Н., Жерлов А.К.  // ДАН, Т.~343, N~5, 1995, с.~583---585.

\bibitem{GilbertAkkerman} Гильберт Д., Аккерман В. Основы теоретической логики. Государственное издательство иностранной литературы. 1947.

\bibitem{Gulamov} Гулямов Ш.Б. Математическое и программное обеспечение решения первопорядковых логических уравнений. канд. дисс. текст. ИДСТУ СО РАН, Иркутск, 1997. 155~С.

\bibitem{yodan} Йодан Э. Структурное проектирование и конструирование программ. 1979. 415с.

\bibitem{QUANT4} Ларионов А.А., Терехин И.Н., Черкашин Е.А., Давыдов А.В.
Программная система КВАНТ/4 для автоматического доказательства теорем
// Труды ИМЭИ ИГУ. Математика и информатика : сб.научных трудов / под
ред.: Ю. Д. Корольков [и др.]. - Иркутск : Изд-во ИГУ, 2011. - Выпуск
1 с. 77-85.

\bibitem{LogicComp} Макаров И.М., и др. Логика и компьютер. Выпуск 5. 2004.

\bibitem{Maslov_1964} Маслов С. Ю. Обратный метод установления выводимости в классическом исчислении предикатов. / Маслов С. Ю. // ДАН СССР, Т.~159, N.~1. 1964. С.~17---20.

\bibitem{NNN} Непейвода Н.Н. Прикладная логика / Издательство Новосибирского Университета. 2000. 524c.

\bibitem{ChenLi} Чень Ч., Ли Р. Математическая логика и автоматическое доказательство теорем. 1983. 360c.

\bibitem{dissChe} Черкашин Е.А. Программная система КВАНТ/1 для автоматического доказательства теорем. канд. дисс. текст. / Черкашин Е.А. / ИДСТУ СО РАН, Иркутск, 1999. 155~С.

\bibitem{Che2} Черкашин Е.А. Разделяемые структуры данных в системе автоматического доказательства теорем КВАНТ/3. / Черкашин Е.А. // Вычислительные технологии. 2008. Т.~13. С.~102---107.

\bibitem{Church1} Чёрч А. Введение в математическую логику. Том 1. 1956 г.




\bibitem{DPL2} Alexandrescu A. The D Programming Language. / Addison-Wesley Professional; 1 edition, June 12, 2010

\bibitem{ATP_DB} Aravindan C., Baumgartner P. Theorem Proving Techniques for View Deletion in Databases // Journal of Symbolic Computation - Special issue on advances in first-order theorem proving. Volume 29 Issue 2, Feb. 2000. pp.~119---147.

\bibitem{erlang} Armstrong J. Programming Erlang. The Pragmatic Programmers. 2007.

\bibitem{Nipkow} Baader, F., Nipkow, T.: Term Rewriting and All That. Cambridge University Press, Cambridge. 1988.

\bibitem{KBAlg} Bachmair L., Dershowitz N., Plaisted D. A. Completion without failure. 1989.

\bibitem{SourceLCL} Balsiger P., Heuerding A., Schwendimann S. A Benchmark Method for the Propositional Modal Logics K, KT, S4 // Journal of Automated Reasoning Volume 24, Number 3 (2000), 297-317.

\bibitem{ProblemOrientedATP1} Bibel W., Korn D., Kreitz C., Schmitt S. Problem-Oriented Applications of Automated Theorem Proving // Design and Implementation of Symbolic Computation Systems. International Symposium, DISCO '96 Karlsruhe, Germany, September 18–20, 1996. pp.~1---21.

\bibitem{ATP_NLP} Blackburn P. Automated Theorem Proving for Natural Language Processing

\bibitem{ATP_Vision} Bondyfalat D., Mourrain B., Papadopoulo T. An Application of Automatic Theorem Proving in Computer vision // Automated Deduction in Geometry. Second International Workshop, ADG’98 Beijing, China, August 1–3, 1998. pp. 207---232.

\bibitem{Bourbaki} Bourbaki N.. Theory of Sets. Paris, Hermann. 1968.


\bibitem{Vassilyev_1997} Butyrin S. A. An {E}xpert {S}ystem for {D}esign of {S}pacecraft  {A}ltitude {C}ontrol {S}ystem. / Butyrin S. A., Makarov V. P., Mukumov R. R., Somov Ye., Vassilyev S. N. // Artificial Intelligence in Engineering. V.~11(1). 1997. P.~49---59.

\bibitem{ATP_Flow} Darvas Á., Hähnle R., Sands D. A Theorem Proving Approach to Analysis of Secure Information Flow // SPC'05 Proceedings of the Second international conference on Security in Pervasive Computing. pp.~193---209.

\bibitem{Frege} Frege G. Begriffsschrift: eine der arithmetischen nachgebildete Formelsprache des reinen Denkens. Halle, 1879.

\bibitem{Ulman} Garcia-Molina H., Ullman J. D., Widom J. Database Systems: The Complete Book / Prentice Hall. 2009. 1210p.

\bibitem{Godel1929} Godel K. Über die Vollständigkeit des Logikkalküls. Doctoral dissertation. 1929.

\bibitem{subtree} Graf P. Substitution Tree Indexing // Proceedings of the 6th International Conference on Rewriting Techniques and Applications. 1995. P.~117---131.

\bibitem{TermIndexingBook} Graf P. Term Indexing (Lecture Notes in Computer Science / Lecture Notes in Artificial Intelligence). Springer; 1 edition. March 27, 1996.

\bibitem{PracticalLogic} Harrison J. Handbook of Practical Logic and Automated Reasoning. Cambridge University Press; 1 edition. April 13, 2009.


\bibitem{SourceBook} Heijenoort J. From Frege to Gödel: A Source Book in Mathematical Logic, 1879-1931, 1967 (1999).

\bibitem{med1} Hommersom A.J., Lucas P.J.F., Bommel P. Automated Theorem Proving for Quality-checking Medical Guidelines // CADE-20 : 20th International Conference on Automated Deduction, Tallinn, Estonia, July 22-27, 2005 : Workshop on Empirically Successful Classical Automated Reasoning (ESCAR).


\bibitem{ATP_geometry} Kortenkamp U., Richter-Gebert J. Using Automatic Theorem Proving to Improve the Usability of Geometry Software // In Proceedings of the Mathematical User Interfaces Workshop. 2004.


\bibitem{PSETHEO}  Letz R., Stenz G., Wolf A. P-SETHEO: Strategy Parallel Automated Theorem Proving // Proceedings of the International Conference on Automated Reasoning with Analytic Tableaux and Related Methods (TABLEAUX-98). 1988.

\bibitem{ATP_Ver2} Maus S., Moskal M., Schulte W. Vx86: x86 Assembler Simulated in C Powered by Automated Theorem Proving // Algebraic Methodology and Software Technology. 12th International Conference, AMAST 2008 Urbana, IL, USA, July 28-31, 2008. pp.~284---298.

\bibitem{disctree} McCune W. Experiments with Discrimination-Tree indexing and Path Indexing for Term Retrieval. / McCune W. W. // Journal of Automated Reasoning. 1992. V.~9(2). P.~147-167.

\bibitem{McCuneRob} McCune W. Solution of the Robbins Problem / McCune W. // Journal of Automated Reasoning, V.~19(3), p.~263---276, December 1997.

\bibitem{microsoft} Moura, L., Bjorner, N. Efficient E-matching for SMT Solvers. Proceedings of th 21st international conference on Automated Deduction: Automated Deduction, Bremen, Germany July 17--20, 2007) 167--182.

\bibitem{Newell1} Newell A., Shaw J.C. Programming the Logic Theory Machine // In Proceedings of the 1957 Western Joint Computer Conference, IRE, 1957, P.230---240.

\bibitem{Newell2} Newell A., Shaw J.C., Simon H.A. Empirical Exploration With the Logic Theory Machine // Proceedings of the Western Joint Computer Conference, Vol. 15, 1957, P.218---239.

\bibitem{constrgeo} Plato J. A Constructive Theory of Ordered Affine Geometry // Indagationes Mathematicae. Volume 9, Issue 4, 21 December 1998. pp.~549---562

\bibitem{Ryazanov2003} Riazanov A. Implementing an Efficient Theorem Prover. / Riazanov A. /  PhD thesis, The University of Manchester, 2003. 216~P.

\bibitem{Robinson_1965} Robinson J. A. A Machine--Oriented Logic Based on the Resolution Principle. / Robinson J. A. //  Journal of the ACM. (12). 1965. P.~23---41.

\bibitem{Eprover} Schulz S. E --- A Brainiac Theorem Prover // AI Communications - CASC. Volume 15 Issue 2,3, August 2002. pp.~111---126.

\bibitem{HARIndex} Sekar R., Ramakrishnan I.V., Voronkov A. Term Indexing. In: Robinson, A., Voronkov, A. (eds.) Handbook of Automated Reasoning, pp. 1853--1964. MIT Press, Cambridge. 2001.

\bibitem{Sourcebook} Smith D.E. A Source Book in Mathematics. 1984.

\bibitem{progress19972001} Sutcliffe G., Fuchs M., Suttner C. Progress in Automated Theorem Proving, 1997-1999 // Workshop on Empirical Methods in Artificial Intelligence, 17th International Joint Conference on Artificial Intelligence, (Seattle, USA), 2001. 53---60.

\bibitem{BTPStickel} Stikel M.E. Building theorem provers // Automated Deduction – CADE-22. 22nd International Conference on Automated Deduction, Montreal, Canada, August 2-7, 2009. pp.~306---321.

\bibitem{pathindex} Stickel M. The path-indexing method for indexing terms. / Stickel M. // Technical Note 473, Artificial Intelligence Center, SRI International, RAVENSWOOD AVE., MENLO PARK, CA 94025

\bibitem{SNV1990} Vassilyev, S.N. Machine Synthesis of Mathematical Theorems. The Journal of Logic programming, V.9, No.2--3, 1990. pp. 235---266

\bibitem{WangHao} Wang H. Toward Mechanical Mathematics // IBM J. Res. Devel., Vol. 4, No 1, 1960, P. 2---22.

\bibitem{PrinMat} Whitehead A.N., Russell B. Principia Mathematica. Cambridge University Press, Cambridge, 2nd edition. January 2, 1927.

\bibitem{Wos_1992} Wos L. Automated Reasoning: Introduction and Applications. / Wos L., Overbeek R., Lusk E.,  Boyle J. / McGraw--Hill,  New York, 19

%----url----

\bibitem{ACL2} ACL2 Version 5.0. URL: http://www.cs.utexas.edu/users/moore/acl2/

\bibitem{DPL1} D Programming Language. URL: http://www.digitalmars.com/d/

\bibitem{Isabelle} Isabelle. URL: http://isabelle.in.tum.de/

\bibitem{keyproj} The KeY Project. URL: http://www.key-project.org/

\bibitem{ModalProver} MOLTAP — A Modal Logic Tableau Prover. URL: http://twan.home.fmf.nl/moltap/

\bibitem{ontobox} Ontobox. URL: http://ontobox.org/

\bibitem{CASC} The CADE ATP System Competition. The World Championship for Automated Theorem Proving. URL: http://www.cs.miami.edu/~tptp/CASC/

\bibitem{LaCoq} The Coq Proof Assistant. URL: http://coq.inria.fr/

\bibitem{HOLprover} The HOL Theorem Prover for Higher Order Logic. URL: http://www.cs.ox.ac.uk/tom.melham/res/hol.html

\bibitem{tptp} Thousands of Problems for Theorem Provers. URL: http://tptp.org/

\bibitem{TPTPTrans} A.V.Gelder. TPTPparser utility. 2006. URL: http://users.soe.ucsc.edu/\~{}avg/TPTPparser/ (access date -- 17.10.2012)

%------

%\bibitem{WangHao_1960} Wang H. Toward Mechanical Mathematics. / Wang H. // IBM J. Res. Devel. V.~4(1), 1960, P.~2---22.

%\bibitem{HeurTP} Roberto Cordeschi. The role of heuristics in automated theorem proving. J.A. Robinson's resolution principle. 1996

%\bibitem{AMDVer} AMD Verification. 2002

%???

%bibitem{Float_Ver} J. Harrison. Float-point verification // International Journal Of Man-Machine Studies. 1995.

%\bibitem{Origami} 2007

\end{thebibliography}



%%% Local Variables:
%%% mode: latex
%%% TeX-master: "dis"
%%% End:
