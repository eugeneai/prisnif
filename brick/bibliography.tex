\begin{thebibliography}{99}

%\bibliography{intelle}
%\bibliographystyle{unsrtre}


\bibitem{ICDS2000} Васильев С.Н. Интеллектное управление динамическими системами. / Васильев С.Н., Жерлов А.К., Федунов Е.А., Федосов Б.Е.  M.:Физматлит, 2000.

\bibitem{Vas1995} Васильев С.Н. Об исчислениях типово-кванторных формул / Васильев С.Н., Жерлов А.К.  // ДАН, Т.~343, N~5, 1995, с.~583-585. 

\bibitem{SNV1990} Vassilyev, S.N.: Machine Synthesis of Mathematical Theorems. The Journal of Logic programming, V.9, No.2--3, pp. 235--266 (1990)

\bibitem{tptp} Thousands of Problems for Theorem Provers {\tt http://tptp.org/}

\bibitem{DPL1} D Programming Language {\tt http://www.digitalmars.com/d/}

\bibitem{DPL2} Alexandrescu A. The D Programming Language. / Addison-Wesley Professional; 1 edition, June 12, 2010 

\bibitem{Maslov_1964} Маслов С. Ю. Обратный метод установления выводимости в классическом исчислении предикатов. / Маслов С. Ю. // ДАН СССР, Т.~159, N.~1. 1964. С.~17--20.

\bibitem{dissChe} Черкашин Е.А. Программная система КВАНТ/1 для автоматического доказательства теорем. канд. дисс. текст. / Черкашин Е.А. / ИДСТУ СО РАН, Иркутск, 1999. 155~С.

\bibitem{Che2} Черкашин Е.А. Разделяемые структуры данных в системе автоматического доказательства теорем КВАНТ/3. / Черкашин Е.А. // Вычислительные технологии. 2008. Т.~13. С.~102-107.

\bibitem{Vassilyev_1997} Butyrin S. A. An {E}xpert {S}ystem for {D}esign of {S}pacecraft  {A}ltitude {C}ontrol {S}ystem. / Butyrin S. A., Makarov V. P., Mukumov R. R., Somov Ye., Vassilyev S. N. // Artificial Intelligence in Engineering. V.~11(1). 1997. P.~49--59.

\bibitem{subtree} Graf P. Substitution Tree Indexing. / Graf P. // Proceedings of the 6th International Conference on Rewriting Techniques and Applications. 1995. P.~117-131.

\bibitem{disctree} McCune W. W. Experiments with Discrimination-Tree indexing and Path Indexing for Term Retrieval. / McCune W. W. // Journal of Automated Reasoning. 1992. V.~9(2). P.~147-167.

\bibitem{McCuneRob} McCune W. Solution of the Robbins Problem / McCune W. // Journal of Automated Reasoning, V.~19(3), p.~263--276, December 1997.

\bibitem{Robinson_1965} Robinson J. A. A Machine--Oriented Logic Based on the Resolution Principle. / Robinson J. A. //  Journal of the ACM. (12). 1965. P.~23--41.

\bibitem{Ryazanov2003} Riazanov A. Implementing an Efficient Theorem Prover. / Riazanov A. /  PhD thesis, The University of Manchester, 2003. 216~P.

\bibitem{pathindex} Stickel M. The path-indexing method for indexing terms. / Stickel M. // Technical Note 473, Artificial Intelligence Center, SRI International, RAVENSWOOD AVE., MENLO PARK, CA 94025

\bibitem{WangHao_1960} Wang H. Toward Mechanical Mathematics. / Wang H. // IBM J. Res. Devel. V.~4(1), 1960, P.~2--22.

\bibitem{Wos_1992} Wos L. Automated Reasoning: Introduction and Applications. / Wos L., Overbeek R., Lusk E.,  Boyle J. / McGraw--Hill,  New York, 19

%------
\bibitem{SourceBook} Jean van Heijenoort / From Frege to Gödel: A Source Book in Mathematical Logic, 1879-1931, 1967 (1999).

\bibitem{PracticalLogic} John Harrison / Handbook of Practical Logic and Automated Reasoning. Cambridge University Press; 1 edition (April 13, 2009)

\bibitem{TermIndexingBook} Peter Graf / Term Indexing (Lecture Notes in Computer Science / Lecture Notes in Artificial Intelligence). Springer; 1 edition (March 27, 1996)


\bibitem{progress19972001} Geoff Sutcliffe / Progress in Automated Theorem Proving

\bibitem{HeurTP} Roberto Cordeschi / The role of heuristics in automated theorem proving.
J.A. Robinson's resolution principle. 1996

\bibitem{ProblemOrientedATP1} W. Bibel, D. Korn, C. Kreitz, and S. Schmitt / Problem-Oriented Applications of Automated Theorem Proving.

\bibitem{PSETHEO} Reinhold Letz / P-SETHEO: Strategy Parallel Automated Theorem Proving

\bibitem{AMDVer} AMD Verification. 2002

\bibitem{ATP_NLP} Patric Blackburn / Automated Theorem Proving for Natural Language Processing

\bibitem{ATP_Vision} Didier Bondyfalat / An Application of Automatic Theorem Proving in Computer vision. 1998

\bibitem{ATP_DB} Chandrabose Aravindan / Theorem Proving Techniques for View Deletion in Databases. 1999

\bibitem{ATP_Flow} Adam Darvas / A Theorem Proving Approach to Analysis of Secure Information Flow 

\bibitem{ATP_Ver2} Stefan Maus / Vx86: x86 Assembler Simulated in C Powered by Automated Theorem Proving. 2008

\bibitem{ATP_geometry} Ulrich Kortenkamap / Using Automatic Theorem Proving to Improve the Usability of Geometry Software. 2004

\bibitem{Origami} 2007

\bibitem{Float_Ver} Harison. Float. Minimum 2002

\bibitem{Butakov1} Бутаков, Курганский.

\bibitem{ChenLi} Чень, Ли. Математическая логика и автоматическое доказательство теорем.

\bibitem{HARIndex} Sekar, R., Ramakrishnan, I.V., Voronkov, A.: Term Indexing. In: Robinson, A., Voronkov, A. (eds.) Handbook of Automated Reasoning, pp. 1853--1964. MIT Press, Cambridge (2001)

\bibitem{Nipkow} Baader, F., Nipkow, T.: Term Rewriting and All That. Cambridge University Press, Cambridge (1988)

\bibitem{Bourbaki} Bourbaki, N.: Theory of Sets. Paris, Hermann (1968)

\bibitem{microsoft} Moura, L., Bjorner, N.: Efficient E-matching for SMT Solvers. Proceedings of th 21st international conference on Automated Deduction: Automated Deduction, Bremen, Germany July 17--20, 2007) 167--182.

\bibitem{BTPStickel} Stikel, M.E.: Building theorem provers (2010)

\bibitem{Eprover} Schulz, S.: E --- A Brainiac Theorem Prover.

\bibitem{Frege} Frege G. Begriffsschrift: eine der arithmetischen nachgebildete Formelsprache des reinen Denkens. Halle, 1879.

\bibitem{Sourcebook} Smith D.E. A Source Book in Mathematics. 1984.

\bibitem{Godel1929} Godel K. Über die Vollständigkeit des Logikkalküls. Doctoral dissertation. 1929.

\bibitem{GilbertAkkerman} Гильберт Д., Аккерман В. Основы теоретической логики. Государственное издательство иностранной литературы. 1947.

\bibitem{LogicComp} Логика и компьютер. Выпуск 5. 2004.

\bibitem{PrinMat} Whitehead A.N., Russell B. Principia Mathematica. Cambridge University Press, Cambridge, 2nd edition (January 2, 1927).

\bibitem{Newell1} Newell A., Shaw J.C. Progrmaiing the Logic Theory Machine // In Proceedings of the 1957 Western Joint Computer Conference, IRE, 1957, P.230-240.

\bibitem{Newell2} Newell A., Shaw J.C., Simon H.A. Empirical Exploration With the Logic Theory Machine // Proceedings of the Western Joint Computer Conference, Vol. 15, 1957, P.218-239.

\bibitem{WangHao} Wang Hao. Toward Mechanical Mathematics // IBM J. Res. Devel., Vol. 4, No 1, 1960, P. 2-22. 

\bibitem{Bourbaki} Bourbaki, N.: Theory of Sets. Paris, Hermann (1968)

\bibitem{NNN} Непейвода Н.Н. Прикладная логика.

\bibitem{Bratko} Братко И. Программирование для ИИ на языке Пролог.

\bibitem{Ulman} Database Systems: The Complete Book. 2000.

\bibitem{KBAlg} Ian Wehrman, Completion without failure.

\end{thebibliography}



%%% Local Variables: 
%%% mode: latex
%%% TeX-master: "dis"
%%% End: 
