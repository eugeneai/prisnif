\begin{thebibliography}{99}

%\bibliography{intelle}
%\bibliographystyle{unsrtre}

\bibitem{Bratko} Братко И. Алгоритмы искусственного интеллекта на языке PROLOG / Издательство Вильямс. 2004. 640с.

\bibitem{Butakov1} Бутаков М.И., Курганский В.И. О решении задачи трансляции планированием вычислений на основе позитивно--образованных формул // Системы управления и информационные технологии. 2007. c.~120-123.

\bibitem{ICDS2000} Васильев С.Н., Жерлов А.К., Федунов Е.А., Федосов Б.Е.  Интеллектное управление динамическими системами /  M.:Физматлит, 2000.

\bibitem{Vas1995} Васильев С.Н. Об исчислениях типово-кванторных формул / Васильев С.Н., Жерлов А.К.  // ДАН, Т.~343, N~5, 1995, с.~583---585.

\bibitem{GilbertAkkerman} Гильберт Д., Аккерман В. Основы теоретической логики. Государственное издательство иностранной литературы. 1947.

\bibitem{Gulamov} Гулямов Ш.Б. Математическое и программное обеспечение решения первопорядковых логических уравнений. канд. дисс. текст. ИДСТУ СО РАН, Иркутск, 1997. 155~С.

\bibitem{DavydovX} Давыдов А.В. Исчисление позитивно-образованных формул с функциональными символами // Прикладные алгоритмы в дискретном анализе: сб. науч. тр. /Под ред. Д-ра физ.-мат. Наук, проф. Ю.Д. Королькова. – Иркутск : Изд-во Иркут. гос. Ун-та, 2008. – 157 с. – (Дискретный анализ и информатика, Вып. 2). C. 23-49.


\bibitem{semver} Давыдов А.В., Ларионов А.А. Об исчислении позитивно-образованных формул для автоматического доказательства теорем. / Тр. 5-го международного симпозиума по компьютерным наукам в России. Семинар ``Семантика, спецификация и верификация программ: теория и приложения''. 14-15 июня 2010, Казань, с. 109-116.

\bibitem{mais} Давыдов А.В., Ларионов А.А., Черкашин Е.А. Об исчислении позитивно-образованных формул для автоматического доказательства теорем. // Моделирование и анализ информационных систем. 2010.T. 17, N 4, С. 60--69.



\bibitem{yodan} Йодан Э. Структурное проектирование и конструирование программ. 1979. 415с.

\bibitem{QUANT4} Ларионов А.А., Терехин И.Н., Черкашин Е.А., Давыдов А.В. Программная система КВАНТ/4 для автоматического доказательства теорем // Труды ИМЭИ ИГУ. Математика и информатика : сб.научных трудов / под ред.: Ю. Д. Корольков [и др.]. - Иркутск : Изд-во ИГУ, 2011. - Выпуск 1 с. 77-85.

\bibitem{distvirt} Ларионов А.А., Черкашин Е.А. Параллельные схемы алгоритмов автоматического доказательства теорем в исчислении позитивно-образованных формул. // Дистанционное и виртуальное обучение. Февраль 2012. No. 2, С. 93-100.

\bibitem{viner} Ларионов А.А., Черкашин Е.А., Давыдов А.В. Программная система для автоматического доказательства теорем в исчислении позитивно-образованных формул. / Винеровские чтения / Труды IV Всероссийской конференции. Часть II. - Иркутск: ИрГТУ, 2011. с.190--197.

\bibitem{burvest} Ларионов А.А., Черкашин Е.А., Терехин И.Н. Системные предикаты для управления логическим выводом в системе автоматического доказательства теорем для исчисления позитивно-образованых формул. // Вестник Бурятского государственного университета, 2011, выпуск 9. Серия Математика. Информатика. с. 94-98.

\bibitem{LogicComp} Макаров И.М., и др. Логика и компьютер. Выпуск 5. 2004.

\bibitem{mendelson} Мендельсон Э. Введение в математическую логику.

\bibitem{Maslov_1964} Маслов С. Ю. Обратный метод установления выводимости в классическом исчислении предикатов. / Маслов С. Ю. // ДАН СССР, Т.~159, N.~1. 1964. С.~17---20.

\bibitem{NNN} Непейвода Н.Н. Прикладная логика / Издательство Новосибирского Университета. 2000. 524c.

\bibitem{ChenLi} Чень Ч., Ли Р. Математическая логика и автоматическое доказательство теорем. 1983. 360c.

\bibitem{dissChe} Черкашин Е.А. Программная система КВАНТ/1 для автоматического доказательства теорем. канд. дисс. текст. / Черкашин Е.А. / ИДСТУ СО РАН, Иркутск, 1999. 155~С.

\bibitem{Che2} Черкашин Е.А. Разделяемые структуры данных в системе автоматического доказательства теорем КВАНТ/3. / Черкашин Е.А. // Вычислительные технологии. 2008. Т.~13. С.~102---107.

\bibitem{Church1} Чёрч А. Введение в математическую логику. Том 1. 1956 г.




\bibitem{DPL2} Alexandrescu A. The D Programming Language. / Addison-Wesley Professional; 1 edition, June 12, 2010

\bibitem{ATP_DB} Aravindan C., Baumgartner P. Theorem Proving Techniques for View Deletion in Databases // Journal of Symbolic Computation - Special issue on advances in first-order theorem proving. Volume 29 Issue 2, Feb. 2000. pp.~119---147.

\bibitem{erlang} Armstrong J. Programming Erlang. The Pragmatic Programmers. 2007.

\bibitem{Nipkow} Baader, F., Nipkow, T.: Term Rewriting and All That. Cambridge University Press, Cambridge. 1988.

\bibitem{KBAlg} Bachmair L., Dershowitz N., Plaisted D. A. Completion without failure. 1989.

\bibitem{SourceLCL} Balsiger P., Heuerding A., Schwendimann S. A Benchmark Method for the Propositional Modal Logics K, KT, S4 // Journal of Automated Reasoning Volume 24, Number 3 (2000), 297-317.

\bibitem{Beth1955} Beth E. W., Semantic entailment and formal derivability / Mededlingen van de Koninklijke Nederlandse Akademie van Wetenschappen, Afdeling Letterkunde, N.R. Vol 18, no 13, 1955, pp 309–42. Reprinted in Jaakko Intikka (ed.) The Philosophy of Mathematics, Oxford University Press, 1969.


\bibitem{ProblemOrientedATP1} Bibel W., Korn D., Kreitz C., Schmitt S. Problem-Oriented Applications of Automated Theorem Proving // Design and Implementation of Symbolic Computation Systems. International Symposium, DISCO '96 Karlsruhe, Germany, September 18–20, 1996. pp.~1---21.

\bibitem{ATP_NLP} Blackburn P. Automated Theorem Proving for Natural Language Processing

\bibitem{ATP_Vision} Bondyfalat D., Mourrain B., Papadopoulo T. An Application of Automatic Theorem Proving in Computer vision // Automated Deduction in Geometry. Second International Workshop, ADG’98 Beijing, China, August 1–3, 1998. pp. 207---232.

\bibitem{Bourbaki} Bourbaki N.. Theory of Sets. Paris, Hermann. 1968.

\bibitem{protolan} Brak R.L., Fleuriot, J.D., McGinnis, J. Theorem proving for protocol languages // In: The Second European Workshop on Multi-Agent Systems, Barcelona, Spain. 2004.

\bibitem{Vassilyev_1997} Butyrin S. A., Makarov V. P., Mukumov R. R., Somov Ye., Vassilyev S. N. An {E}xpert {S}ystem for {D}esign of {S}pacecraft  {A}ltitude {C}ontrol {S}ystem // Artificial Intelligence in Engineering. V.~11(1). 1997. P.~49---59.

\bibitem{chellas} Chellas B.F. Modal Logic: An Introduction / Cambridge University Press. February 29, 1980. 312p.

\bibitem{hardver} Claessen K., Hähnle R., Mårtensson J. Verification of Hardware Systems with First-Order Logic //

\bibitem{tableau2} D'Agostino M., Gabbay Dov M., Hähnle R., Posegga J., eds. Handbook of Tableau Methods /  Kluwer. Academic Publishers, Dordrecht 1999. 680p.

\bibitem{ATP_Flow} Darvas Á., Hähnle R., Sands D. A Theorem Proving Approach to Analysis of Secure Information Flow // SPC'05 Proceedings of the Second international conference on Security in Pervasive Computing. pp.~193---209.

\bibitem{accs} Davydov A.V., Larionov A.A., Cherkashin E.A. On the calculus of positively constructed formulas for automated theorem proving // Automatic Control and Computer Sciences (AC\&CS). 2011. Volume 45, Issue 7, pp.~402-407.

\bibitem{inverse} Degtyarev A., Voronkov A. The Inverse Method. In: Robinson, A., Voronkov, A. (eds.) Handbook of Automated Reasoning, pp. 181--270. MIT Press, Cambridge. 2001.


\bibitem{Frege} Frege G. Begriffsschrift: eine der arithmetischen nachgebildete Formelsprache des reinen Denkens. Halle, 1879.

\bibitem{Ulman} Garcia-Molina H., Ullman J. D., Widom J. Database Systems: The Complete Book / Prentice Hall. 2009. 1210p.

\bibitem{gentzen1935} Untersuchungen uber das Logische Schliessen / Mathematische Zeitschrift 39. pp.~176---209, 405---431.
%English translation \cite{Szabo1969}


\bibitem{Godel1} Godel K. Die Vollstandigkeit der Axiome des Logischen Funktionenkalkuls. - Monatsh. Math. Phys., 1930, 37, pp. 349--360.

\bibitem{Godel2} Godel K. Uber formal unentscheidbare Satze der Principia Mathematica und verwandfer Systeme, I. - Monatsh. Math. Phys., 1931, 38, pp. 173--198.



\bibitem{subtree} Graf P. Substitution Tree Indexing // Proceedings of the 6th International Conference on Rewriting Techniques and Applications. 1995. P.~117---131.

\bibitem{TermIndexingBook} Graf P. Term Indexing (Lecture Notes in Computer Science / Lecture Notes in Artificial Intelligence). Springer; 1 edition. March 27, 1996.

\bibitem{PracticalLogic} Harrison J. Handbook of Practical Logic and Automated Reasoning. Cambridge University Press; 1 edition. April 13, 2009.


\bibitem{SourceBook} Heijenoort J. From Frege to Gödel: A Source Book in Mathematical Logic, 1879-1931, 1967 (1999).

\bibitem{med1} Hommersom A.J., Lucas P.J.F., Bommel P. Automated Theorem Proving for Quality-checking Medical Guidelines // CADE-20 : 20th International Conference on Automated Deduction, Tallinn, Estonia, July 22-27, 2005 : Workshop on Empirically Successful Classical Automated Reasoning (ESCAR).


\bibitem{korovin9} Korovin K., Podelski A., Voronkov A., Wilhelm R. An Invitation to Instantiation-Based Reasoning: From Theory to Practice // Volume in Memoriam of Harald Ganzinger, 163-166, Lecture Notes in Computer Science 5663, Springer-Verlag.

\bibitem{ATP_geometry} Kortenkamp U., Richter-Gebert J. Using Automatic Theorem Proving to Improve the Usability of Geometry Software // In Proceedings of the Mathematical User Interfaces Workshop. 2004.


\bibitem{mipro}  Larionov A.A., Cherkashin E.A., Davydov A.V. Theorem Proving Software, Based on Method of Positively-Constructed Formulae // MIPRO 2011. 34-th international convention on information and communication technology, electronics and microelectronics. Vol. III. May 23-27, 2011. Croatia, Opatija. //MIPRO: Croatia. - 2011., pp. 365--368.

\bibitem{tableau1} Letz R., Stenz G. Model Elimination and Connection Tableau Procedures, Robinson A., Voronkov A., Handbook of Automated Reasoning. Elsevier Science, 2001. pp.~2015-2114.

\bibitem{PSETHEO}  Letz R., Stenz G., Wolf A. P-SETHEO: Strategy Parallel Automated Theorem Proving // Proceedings of the International Conference on Automated Reasoning with Analytic Tableaux and Related Methods (TABLEAUX-98). 1988.

\bibitem{ATP_Ver2} Maus S., Moskal M., Schulte W. Vx86: x86 Assembler Simulated in C Powered by Automated Theorem Proving // Algebraic Methodology and Software Technology. 12th International Conference, AMAST 2008 Urbana, IL, USA, July 28-31, 2008. pp.~284---298.

\bibitem{disctree} McCune W. Experiments with Discrimination-Tree indexing and Path Indexing for Term Retrieval // Journal of Automated Reasoning. 1992. V.~9(2). P.~147-167.

\bibitem{McCuneRob} McCune W. Solution of the Robbins Problem // Journal of Automated Reasoning, V.~19(3), p.~263---276, December 1997.

\bibitem{microsoft} Moura, L., Bjorner, N. Efficient E-matching for SMT Solvers. Proceedings of th 21st international conference on Automated Deduction: Automated Deduction, Bremen, Germany July 17--20, 2007) 167--182.

\bibitem{Newell1} Newell A., Shaw J.C. Programming the Logic Theory Machine // In Proceedings of the 1957 Western Joint Computer Conference, IRE, 1957, P.230---240.

\bibitem{Newell2} Newell A., Shaw J.C., Simon H.A. Empirical Exploration With the Logic Theory Machine // Proceedings of the Western Joint Computer Conference, Vol. 15, 1957, P.218---239.

\bibitem{constrgeo} Plato J. A Constructive Theory of Ordered Affine Geometry // Indagationes Mathematicae. Volume 9, Issue 4, 21 December 1998. pp.~549---562

\bibitem{promsky} Promsky A.V. Towards C-light Program Verification: Overcoming the Obstacles // Proc. International Workshop on Program Understanding, 19-23 June, Altai Mountains, Russia, 2009. pp.~53---63.

\bibitem{Ryazanov2003} Riazanov A. Implementing an Efficient Theorem Prover /  PhD thesis, The University of Manchester, 2003. 216~P.

\bibitem{Robinson_1965} Robinson J. A. A Machine--Oriented Logic Based on the Resolution Principle. / Robinson J. A. //  Journal of the ACM. (12). 1965. P.~23---41.

\bibitem{HAR} Robinson, A., Voronkov, A. (eds.) Handbook of Automated Reasoning, MIT Press, Cambridge. 2001.

\bibitem{Eprover} Schulz S. E --- A Brainiac Theorem Prover // AI Communications - CASC. Volume 15 Issue 2,3, August 2002. pp.~111---126.

\bibitem{atpsoft} Schumann J. M. Automated Theorem Proving in Software Engineering / Springer, 2001. 228p.

\bibitem{HARIndex} Sekar R., Ramakrishnan I.V., Voronkov A. Term Indexing. In: Robinson, A., Voronkov, A. (eds.) Handbook of Automated Reasoning, pp. 1853--1964. MIT Press, Cambridge. 2001.

\bibitem{Smullyan1995} Smullyan R.M. First-Order Logic / Dover Publications, New York. secong corrected edn. 1995. 158p.

\bibitem{Sourcebook} Smith D.E. A Source Book in Mathematics. 1984.

\bibitem{progress19972001} Sutcliffe G., Fuchs M., Suttner C. Progress in Automated Theorem Proving, 1997-1999 // Workshop on Empirical Methods in Artificial Intelligence, 17th International Joint Conference on Artificial Intelligence, (Seattle, USA), 2001. 53---60.

\bibitem{BTPStickel} Stikel M.E. Building theorem provers // Automated Deduction – CADE-22. 22nd International Conference on Automated Deduction, Montreal, Canada, August 2-7, 2009. pp.~306---321.

\bibitem{pathindex} Stickel M. The path-indexing method for indexing terms // Technical Note 473, Artificial Intelligence Center, SRI International, RAVENSWOOD AVE., MENLO PARK, CA 94025

\bibitem{semweb} Tammet T. Extending Classical Theorem Proving for the Semantic Web // In R. Volz, S. Decker, I. Cruz, editors, Proceedings of the First International Workshop on Practical and Scalable Semantic Systems, October 2003.

\bibitem{turing} Turing A.M. On Computable Numbers, with an Application to the Entscheidungsproblem. - Proc.Lond. Math. Soc., Ser. 2, 1936, 42, pp. 230-265, 1936, 43, pp. 544-546.

\bibitem{SNV1990} Vassilyev, S.N. Machine Synthesis of Mathematical Theorems. The Journal of Logic programming, V.9, No.2--3, 1990. pp. 235---266

\bibitem{WangHao} Wang H. Toward Mechanical Mathematics // IBM J. Res. Devel., Vol. 4, No 1, 1960, P. 2---22.

\bibitem{SPASS} Weidenbach C., Dimova D., Fietzke A., Kumar R., Suda M., Wischnewski P. SPASS Version 3.5 // Automated Deduction – CADE-22. 22nd International Conference on Automated Deduction, Montreal, Canada, August 2-7, 2009. Proceedings. pp.~140---145.

\bibitem{PrinMat} Whitehead A.N., Russell B. Principia Mathematica. Cambridge University Press, Cambridge, 2nd edition. January 2, 1927.

\bibitem{Wos_1992} Wos L. Automated Reasoning: Introduction and Applications. / Wos L., Overbeek R., Lusk E.,  Boyle J. / McGraw--Hill,  New York, 19

%----url----

\bibitem{ACL2} ACL2 Version 5.0. URL: http://www.cs.utexas.edu/users/moore/acl2/ (access date -- 17.10.2012)

\bibitem{DPL1} D Programming Language. URL: http://www.digitalmars.com/d/ (access date -- 17.10.2012)

\bibitem{eprover} E Prover. URL: http://www.eprover.org (access date -- 17.10.2012)

\bibitem{EQP} http://www.cs.unm.edu/\~mccune/eqp/ (access date -- 17.10.2012)

\bibitem{gmemory} GLIB: Memory Slices. URL:http://developer.gnome.org/glib/2.34/glib-Memory-Slices.html (access date -- 17.10.2012)

\bibitem{iprover} iProver. http://www.cs.man.ac.uk/~korovink/iprover/ (access date -- 17.10.2012)

\bibitem{Isabelle} Isabelle. URL: http://isabelle.in.tum.de/ (access date -- 17.10.2012)

\bibitem{keyproj} The KeY Project. URL: http://www.key-project.org/ (access date -- 17.10.2012)

\bibitem{leancop} LeanCoP. URL: http://www.leancop.de (access date -- 17.10.2012)

\bibitem{leoprover} Loprover. URL: http://www.ags.uni-sb.de/\~leo/ (access date -- 17.10.2012)

\bibitem{ModalProver} MOLTAP — A Modal Logic Tableau Prover. URL: http://twan.home.fmf.nl/moltap/ (access date -- 17.10.2012)

\bibitem{ontobox} Ontobox. URL: http://ontobox.org/ (access date -- 17.10.2012)

\bibitem{otter} http://www.cs.unm.edu/~mccune/otter/ (access date -- 17.10.2012)

\bibitem{CASC} The CADE ATP System Competition. The World Championship for Automated Theorem Proving. URL: http://www.cs.miami.edu/~tptp/CASC/ (access date -- 17.10.2012)

\bibitem{LaCoq} The Coq Proof Assistant. URL: http://coq.inria.fr/ (access date -- 17.10.2012)

\bibitem{HOLprover} The HOL Theorem Prover for Higher Order Logic. URL: http://www.cs.ox.ac.uk/tom.melham/res/hol.html (access date -- 17.10.2012)

\bibitem{tptp} Thousands of Problems for Theorem Provers. URL: http://tptp.org/ (access date -- 17.10.2012)

\bibitem{TPTPTrans} A.V.Gelder. TPTPparser utility. 2006. URL: http://users.soe.ucsc.edu/\~{}avg/TPTPparser/ (access date -- 17.10.2012)

\bibitem{vprover} Vampire. URL: http://www.vprover.org (access date -- 17.10.2012)

%------

%\bibitem{WangHao_1960} Wang H. Toward Mechanical Mathematics. / Wang H. // IBM J. Res. Devel. V.~4(1), 1960, P.~2---22.

%\bibitem{HeurTP} Roberto Cordeschi. The role of heuristics in automated theorem proving. J.A. Robinson's resolution principle. 1996

%\bibitem{AMDVer} AMD Verification. 2002

%???

%bibitem{Float_Ver} J. Harrison. Float-point verification // International Journal Of Man-Machine Studies. 1995.

%\bibitem{Origami} 2007

\end{thebibliography}



%%% Local Variables:
%%% mode: latex
%%% TeX-master: "dis"
%%% End:
