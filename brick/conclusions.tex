\chapter*{Заключение}

%Общая характеристика полученных в диссертации результатов.
В диссертации представлены результаты исследования исчисления ПО-формул, как формализма для автоматического доказательства теорем (АДТ). Основным результатом является программная система АДТ, в которой реализован ряд стратегий повышающих эффективность логического вывода. Среди предложенных и реалиозованных стратегий имеются как адаптированные из других методов для ПО--формул, так и оригинальные. 

Разработанный вид системы АДТ (пруверов) занимает промежуточное положение между интерактивными системами и классическими, позволя с одной строны производить достаточно автоматизированный логический вывод, с другой стороны вовлекать в процесс человека, тем самым используя основной дубль свойств человеко- и машинно-ориентированность.

Любые исследования в данной области являются актуальными. Формальное доказательство теоремы может иметь сколь угодно большую минимальную длину, поэтому любые методы позволяющие сократить число шагов вывода, а так же скорость их проведения важны. Многие современные и самые производительные системы АДТ уже столкнулись с проблемой сложности расширения и использования дополнительных знаний о задаче (универсализм систем). Данная работа является реализацией одного из способо решения этих проблем.


% Выносимые на защиту результаты.
В рамках диссертации получены и на защиту выносятся следующие результаты:
\begin{enumerate}
\item Адаптированны некоторые общеупотребимые стратегии. Три типа параллельных стратегий, индексирование термов, один вариант разделения памяти (data sharing).
\item Реализоваы оригинальные стратегии предложенные именно для исчисления ПО-формул. Дерево состояний вывода, $k,m$--ограничение, стратегии напарвленыные на решение проблем открытых переменных, варианты разделения памяти.
\item Реализован транслятор формул из языка библиотеки TPTP в язык ПО--формул.
\item Показана хорошая производительнсть системы на задачах из библиотетки TPTP.
\end{enumerate}

% Достигнута ли цель диссертации?
В диссертации показана возможность использования разработанных инструментальных средств при создании программных систем.

%Конструктивная [само]критика результатов диссертации.
Несмотря на то, что в диссертации получен ряд положительных результато, необходимо сделать следующие замечания.

Основными особенностями, препятствующими внедрению разработанных инструментальных средств, на наш взгляд, являются:

\begin{enumerate}
\item Условие достаточно хорошего понимания принципов логического программирования и особенностей формализма ПО--формул у пользователя;
\item Проблема равенств решена не так эффективно, как в иных системах АДТ. %Универсальная слабость прувера. То есть система показывает себя не достаточно сильно при решении задач без  использования эвристик и модификаторов семантик. 
\end{enumerate}

Проблема 1 может быть частично или полностью решена путём развития пользовательских интерфейсов и путем просвящения пользователей. Проблема 2 требует дальнейшего отдельного исследования.

%На проблему 2 стоит смотреть с точки зрения области применения систем. То есть не использовать систему там где необходимо решать универсальным методом, и использовать там где возможно использование сильных сторон исчисления ПО-формул.

% Направления дальнейших исследований и развития в целом.

Перспективы развития естественным образом делятся на два класса: развитие фундаментальной части (языка и исчисления ПО--формул), в частности исследования вопросов построения модальных исчислений, дескриптивных, высшего порядка; и непосрдественно расширение класса решаемых задач, в частности применение в области семантического веба (как достаточно хорошо развивающегося направления), биоинформатики.

Дальнейшая разработка системы связана с продолжнием повышения производительности поиска ЛВ.


%%% Local Variables: 
%%% mode: latex
%%% TeX-master: "dis"
%%% End: 
