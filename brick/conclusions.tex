\chapter*{Заключение}

%Общая характеристика полученных в диссертации результатов.
В диссертации представлены результаты исследования исчисления ПО--формул, как формализма для автоматического доказательства теорем (АДТ). Основным результатом является программная система АДТ, в которой реализован ряд стратегий повышающих эффективность поиска логического вывода. При этом эффективность формализована по следующим критериям: время, количество шагов вывода, количество потребляемой памяти, ширина классов решаемых задач.

Среди предложенных и реализованных стратегий имеются как адаптированные из других методов АДТ для исчисления ПО--формул, так и оригинальные.

Применение системы для решения ряда задач из библиотеки TPTP показало, что система соответствует мировому уровню в данной области. Указаны классы задач, как с точки зрения формы их представления, так с точки зрения предметных областей, на которых разработанная система выигрывает у основных мировых лидеров. Кроме того, решен ряд крайне сложных задач (рейтинг TPTP 1.0).

%Разработанный вид системы АДТ (пруверов) занимает промежуточное положение между интерактивными системами и классическими, позволяя с одной стороны производить достаточно автоматизированный логический вывод, с другой стороны вовлекать в процесс человека, тем самым используя основной дубль свойств человеко-- и машинно--ориентированность.

%Любые исследования в данной области являются актуальными. Формальное доказательство теоремы может иметь сколь угодно большую минимальную длину, поэтому любые методы позволяющие сократить число шагов вывода, а так же скорость их проведения важны. Многие современные и самые производительные системы АДТ уже столкнулись с проблемой сложности расширения и использования дополнительных знаний о задаче (универсализм систем). Данная работа является реализацией одного из способов решения этих проблем.

% Выносимые на защиту результаты.
В рамках диссертации получены и на защиту выносятся следующие результаты:
\begin{enumerate}
\item Новые методки для повышения эффективности поиска ЛВ в исчислении ПО--формул: эффективные структуры представления ПО--формул,  стратегии работы с неограниченными переменными, $k,m$--ограничение, экономия памяти, работа с предикатом равенства без явного использования аксиом равенства.

\item Успешно применены существующие методики повышения эффективности поиска ЛВ для исчисления ПО--формул: индексирование термов, параллельные схемы алгоритмов, разделение данных для термов.

\item Разработана программная система для эффективного поиска логического вывода в исчислении ПО--формул.

\item Для разработанной системы значительно расширен класс успешно решаемых задач, по сравнению с предыдущими системами АДТ базирующимися на исчислении ПО--формул. Разработана инфраструктура взаимодействия с библиотекой задач TPTP. Выделены классы задач, которые система решает эффективнее чем другие системы АДТ. Решены сложнейшие задачи, с максимальным рейтингом TPTP 0.92
\end{enumerate}

Таким образом поставленная цель диссертационной работы достигнута. Необходимые для достижения цели задачи решены.

%\begin{enumerate}
%\item Исследование языка и исчисления ПО--формул, анализ его свойств, влияющих на возможность адаптации существующих алгоритмов и разработки новых алгоритмов;
%\item Разработка эффективных структур данных представления ПО--формул в памяти компьютера;
%\item Разработка алгоритмов преобразования ПО--формул для организации автоматического логического ввода;
%\item Адаптация существующих методик реализации алгоритмов АДТ для исчисления ПО--формул;
%\item Исследование вопросов автоматизации построения эффективного логического вывода с использованием предиката равенства;
%\item Разработка программной системы АДТ, создание инструментальных средств для программирования специальных версий АДТ, ориентированных на определенные классы теоретических и практических задач поиска ЛВ.
%\item Апробация разработанных программных средств в решении тестовых и практических задач.
%\end{enumerate}

% Достигнута ли цель диссертации?
%В диссертации показана возможность использования разработанных инструментальных средств при создании программных систем.

%Конструктивная [само]критика результатов диссертации.
%Несмотря на то, что в диссертации получен ряд положительных результатов, необходимо сделать следующие замечания.

Из особенностей, препятствующих внедрению разработанных инструментальных средств, на наш взгляд, являются:

\begin{enumerate}
\item Условие достаточно хорошего понимания принципов логического программирования и особенностей формализма ПО--формул у пользователя;
\item Проблема равенств решена не так эффективно, как в иных системах АДТ.
\end{enumerate}

Проблема 1 может быть частично или полностью решена путём развития пользовательских интерфейсов и путем просвещения пользователей. Проблема 2 требует дальнейшего отдельного исследования.

% Направления дальнейших исследований и развития в целом.

Перспективы развития естественным образом делятся на два класса: развитие фундаментальной части языка и исчисления ПО--формул, в частности исследования вопросов построения модальных исчислений, дескриптивных, высшего порядка, использования табличного и обратного методов; и непосредственно расширение класса решаемых задач, в частности применение в области семантического веба, биоинформатикии, верификации, открытых математических проблем. 

Дальнейшая разработка системы связана с продолжением повышения эффективности поиска ЛВ.


%%% Local Variables:
%%% mode: latex
%%% TeX-master: "dis"
%%% End:
