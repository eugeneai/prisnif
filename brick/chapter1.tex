\chapter{Теоретический базис работы и постановка задачи}

%-------------------------------------------------------------------
%---------------- ПО-формулы--------------------------------------
\section{Язык и исчисление позитивно-образованных фрмул}

В основе разрабатываемой системы лежит исчисление ПО--формул \cite{ICDS2000}.

Исчисление ПО--формул $\boldsymbol{JF}$ есть тройка $\left\langle \boldsymbol{LF}, Ax\boldsymbol{JF}, \bomega \right\rangle$, где $\boldsymbol{LF}$ --- язык ПО--формул, $Ax\boldsymbol{JF}$ --- единственная схема аксиом и $\bomega$ --- единственное правило вывода $\boldsymbol{JF}.$

%---------------------ЯЗЫК ПО-ФОРМУЛ--------------------
\subsection{Язык позитивно--образованных формул}

Будем обозначать множество всех конъюнктов как $Con$ и положим что {\em конъюнкт} либо конечное множестве обычных атомов языка предикатов первого порядка либо $\boldsymbol{False}$, где $\boldsymbol{False}$ удовлетворяет условию $A \subset \boldsymbol{False} $ для любого $A \in Con$. Пустой конъюнкт обозначается как $\boldsymbol{True}$. Очевидно что если $A \in Con$ тогда $A \cup \boldsymbol{False} = \boldsymbol{False}$. Атомы любого конъюнкта, исключая $\boldsymbol{True}$ и $\boldsymbol{False}$, могут содержать переменные, константные и функциональные символы.

\begin{definition}\label{def:1}
Пусть $\bar{x}$ есть множество переменных и $A$ есть конъюнкт. Правильно-построенные формулы языка ПОФ определяются следующим образом:

Выражение вида $\exists \bar{x}\colon A$ есть $\exists$--формула; выражение вида $\forall \bar{x}\colon A$ есть $\forall$--формула.

Пусть $G_1,\ldots,G_k$ являются $\exists$--формулами, тогда $\forall$--формула имеет следующий вид: $\forall \bar{x}\colon A\left(G_1,\ldots,G_k\right)$.

Пусть $G_1,\ldots,G_k$ являются $\forall$--формулами, тогда $\exists$--формула имеет следующий вид: $\exists \bar{x}\colon A\left(G_1,\ldots,G_k\right)$.

\end{definition}

Выражение является правильно построенной ПО--формулой если оно построено только по правилам Определения \ref{def:1}.

Переменные из $\bar{x}$ связаны соответствующими кванторами и называются $\forall$-переменные и $\exists$-переменные, соответственно.

$\forall$-переменная которая не встречается в соответствующем конъюнкте называется {\em неограниченной} переменной.

%semantic
Теперь определим семантику ПО--формул как семантику соответствующих формул языка предикатов первого порядка.

\begin{definition}\label{def:semantic}
Пусть $A = \{A_1,\ldots,A_l\}$ есть конъюнкт и $\bar{x} = \{x_1,\ldots,x_n\}$ --- множество переменных. Через $A^{\&}$ обозначим $A_1 \&\ldots\&A_l$, при этом $\boldsymbol{False}^{\&}= False, \boldsymbol{True}^{\&}=True$ (пропозициональные константы). Через $F^{\text{\tiny{FOF}}}$ обозначим образ соответствующей ПО--формулы $F$ в языке FOL.

Если $F= \exists \bar{x}\colon A$ то $F^{\text{\tiny{FOF}}} = \exists x_1 \ldots \exists x_n \left(A^{\&}\right).$

Если $F = \forall \bar{x}\colon A$ то $F^{\text{\tiny{FOF}}} = \forall x_1 \ldots \forall x_n \left(A^{\&}\right).$

Если $F = \exists \bar{x}\colon A\left(G_1,\ldots,G_k\right)$ то $F^{\text{\tiny{FOF}}} = \exists x_1 \ldots \exists x_n  \left(A^{\&} \& \left(G_{1}^{\text{\tiny{FOF}}} \& \ldots \& G_{k}^{\text{\tiny{FOF}}}\right)\right).$

Если $F = \forall \bar{x}\colon A\left(G_1,\ldots,G_k\right)$ то $F^{\text{\tiny{FOF}}} = \forall x_1 \ldots \forall x_n \left(A^{\&}\rightarrow \left(G_{1}^{\text{\tiny{FOF}}} \vee\ldots\vee G_{k}^{\text{\tiny{FOF}}}\right)\right).$

\end{definition}

Любая ПО--формула очевидно имеет структуру дерева. Таким образом, для удобства читаемости мы будем представлять их как древовидные структуры а также пользоваться соответствующей терминологии: узел, корень, ветвь, лист и т.д.

Если ПО--формула $F$ начинается с $\forall\colon\boldsymbol{True}$ узла, и каждый лист $F$ является $\exists$--узлом, то $F$ называется ПО--формулой в {\em канонической форме}.
Очевидно что любая ПО--формула $F$ может быть приведена к каноническому виду с помощью следующих преобразований:
\begin{enumerate}
\item Если $F$ не каноническая $\forall$--формула, тогда $\forall\colon \boldsymbol{True}\left(\exists\colon \boldsymbol{True}\left(F\right)\right)$ есть ПО--формула начинающаяся с $\forall\colon\boldsymbol{True}$.
\item Если $F$ есть $\exists$--формула, тогда $\forall\colon \boldsymbol{True}\left(F\right)$ есть ПО--формула начинающаяся с $\forall\colon\boldsymbol{True}$.
\item Если $F$ имеет лист $\forall \bar{x}\colon A$, тогда новый узел $\exists\colon\boldsymbol{False}$ может быть добавлен как потомок.
\end{enumerate}

%В дальнейшем, если не оговорено обратного, будем рассматривать только ПОФы в каноническом виде.

Некоторые части ПО--формул имеют специальные названия: корневой (0-глубины) узел называется {\em корнем} ПО--формулы; любой узел глубины 1 называется {\em базой} ПО--формулы; максимальное поддерево начинающееся с узла глубины 1 называется {\em базовой подформулой}; любой узел глубины 2 называется {\em вопросом} к базе; максимальное поддерево начинающееся с узла глубины 2 называется {\em подформулой-вопросом}; максимальное поддерево начинающееся с узла глубины 3 называется {\em консеквентом}. В соответствии с определением семантики если узел чётной (нечетной) глубины имеет более чем одного потомка, то будем говорить что этот узел имеет {\em дизъюнктивное ветвление} (соответственно {\em конъюнктивное ветвление}).

\begin{example}
Рассмотрим формулу языка FOL
$$F= \neg\bigl(\forall x\:\exists y P(x,y)\rightarrow \exists z P(z,z)\bigr).$$
Образ $F^{\text{\tiny{PCF}}}$ формулы $F$ в языке ПО--формул есть
$$F^{\text{\tiny{PCF}}} = \forall\colon \boldsymbol{True}-\exists\colon\boldsymbol{True} \left\{
\begin{array}{lcl}
 \forall x\colon\boldsymbol{True} & - & \exists y\colon P(x,y) \\
 \forall z\colon P(z,z) & - & \exists\colon\boldsymbol{False}
\end{array}
\right.$$

\end{example}


%-------------------------ИСЧИСЛЕНИЕ ПО-ФОРМУЛ-------------------
\subsection{Исчисление позитивно-образованных формул}

Схема аксиом исчисления ПО--формул $\boldsymbol{JF}$ имеет следующую форму:
$$ Ax\boldsymbol{JF} = \forall\colon\boldsymbol{True}\left(\exists \bar{x}_1\colon\boldsymbol{False}\left(\widetilde{\Phi}_1\right),\ldots,\exists \bar{x}_n\colon\boldsymbol{False}\left(\widetilde{\Phi}_n\right)\right) $$

В исчислении ПО--формул для того что бы доказать $F$ мы будем пытаться опровергнуть её отрицание, поэтому аксиом\app{а} исчисления ПО--формул ---  тождественно ложное высказывание. Таким образом процесс вывода в исчислении ПО--формул является процессом {\em опровержения}.

\begin{definition}
\label{ircond}
Будем говорить что вопрос $\forall \bar{y}\colon A$ к базе $\exists \bar{x}\colon B$ имеет {\em ответ} $\theta$  тогда и только тогда, когда $\theta$ есть подстановка $\bar{y} \rightarrow H^{\infty}$ и $A\theta \subseteq B$, где $H^{\infty}$ есть Эрбранов универсум основанный на $\exists$--переменных из $\bar{x}$, константных и функциональных символах которые встречаются в соответствующей базе.
\end{definition}

%The unique unary inference rule $\bomega$ is defined as follows.

\begin{definition}
\label{omega}
Если $F$ имеет структуру $\forall\colon\boldsymbol{True}\left(\exists \bar{x}\colon B\left(\Phi\right),\Sigma\right)$, где $\Sigma$ есть список других базовых подформул, и $\Phi$ есть список подформул--вопросов содержащий подформулу--вопрос $\forall \bar{y}\colon A(\exists \bar{z}_i\colon C_i\left(\Psi_i\right))_{i=\overline{1,k}}$, тогда заключение $\bomega F$ есть результат применения унарного правила вывода $\bomega$ к вопросу $\forall \bar{y}\colon A$ с ответом $\theta$, и $\bomega F = \forall\colon\boldsymbol{True}(\exists \bar{x} \cup \bar{z}_i\colon B \cup C_i\theta\left(\Phi \cup \Psi_i\theta\right)_{i=\overline{1,k}},\Sigma).$

\end{definition}

После соответствующего переименования некоторых переменных в каждой подформуле, выражение $\bomega F$ будет удовлетворять всем условиям правильно--построенных ПО--формул.

%
Любая конечная последовательность ПО--формул $F, \bomega F, \bomega^2 F,\ldots,\bomega^n F$, где $\bomega^n F \in Ax\boldsymbol{JF}$ называется {\em опровержением} $F$ в исчислении ПО--формул. Иногда будем использовать слово вывод вместо опровержение.

Вопрос с консеквентом $\exists:\boldsymbol{False}$ называется {\em целевым вопросом}. Если вопрос имеет дизъюнктивное ветвление, то ответ на этот вопрос приводит к расщеплению соответствующей базовой подформулы на несколько новых.

Для ясности рассмотрим пример.

\begin{example}[Опровержение в $\boldsymbol{JF}$]\label{proofexample}


\begin{equation*}\label{ex:f1}
	F_1 = \forall\colon\boldsymbol{True} - \exists\colon S(e)(Q_1,Q_2,Q_3,Q_4);
\end{equation*}
\begin{equation*}
	\begin{array}{l}
	Q_1 = \forall x\colon S(x) - \exists\colon A(a) \\
	Q_2 = \forall x,y\colon C(x),D(y) - \exists\colon\boldsymbol{False} \\
	Q_3 = \forall x,y\colon B(x),C(f(y)) - \exists\colon\boldsymbol{False} \\
	Q_4 =
	\forall x\colon A(x) -
	\left\lbrace
	\begin{array}{l}
		\exists y\colon B(y),C(f(x)) \\
		\exists \colon C(x) - \forall z\colon A(z),C(z) - \exists\colon D(f(z))
	\end{array}\right.
	\end{array}
\end{equation*}
На первом шаге вывода существует только один ответ $\{x \rightarrow e\}$ на вопрос $Q_1$. После применения $\bomega$ с этим ответом, формула приобретает следующий вид:
\begin{equation*}\label{ex:f2}
	F_2 = \forall\colon\boldsymbol{True} - \exists\colon S(e),A(a)(Q_1,Q_2,Q_3,Q_4)
\end{equation*}
На втором шаге вывода существует только один ответ $\{x \rightarrow a\}$ на вопрос $Q_4$. После применения $\bomega$ с этим ответом, формула расщепляется, потому что $Q_4$ имеет дизъюнктивное ветвление. И теперь формулы имеет следующий вид:

\begin{equation*}\label{ex:f3}
F_3 =
\forall:\boldsymbol{True} -
\left\lbrace
\begin{array}{l}
	\exists y_1\colon S(e),A(a),B(y_1),C(f(a)) -
	\left\lbrace
	\begin{array}{l}
		Q_1 \\ \cdots \\ Q_4
	\end{array}\right. \\
	\exists\colon S(e)A(a),C(a) -
	\left\lbrace
	\begin{array}{l}
		Q_1 \\ \cdots \\ Q_4 \\
		\forall z\colon A(z),C(z) - \exists\colon D(f(z))
	\end{array}\right. \\
\end{array}\right.
\end{equation*}

На третьем шаге вывода первая база может быть опровергнута ответом $\{x \rightarrow y_1; y \rightarrow a\}$ на целевой вопрос $Q_3$. Опровергнутая база (подформула) для удобства представления может быть удалена из списка базовых подформул.

На четвертом шаге вывода существует ответ $\{z \rightarrow a\}$ на пятый новый вопрос. И формула приобретает следующий вид:
\begin{equation*}\label{ex:f5}
	F_4 = \forall\colon\boldsymbol{True} - \exists\colon S(e),A(a), C(a),D(f(a))
	\left\lbrace
	\begin{array}{l}
		Q_1 \\ \cdots \\ Q_4 \\
		\forall z\colon A(z),C(z) - \exists\colon D(f(z))
	\end{array}\right.
\end{equation*}

На пятом шаге вывода единственная база может быть опровергнута ответом $\{x \rightarrow a; y \rightarrow f(a)\}$ на целевой вопрос $Q_4$.

Опровержение закончено поскольку все базы опровергнуты.

\end{example}

%Корректность и полнота доказаны в \cite{}.

%----------------------------------ОСОБЕННОСТИ ПО-ФОРМУЛ--------------------
\subsection{Особенности ПО--формул}
В литературе выделяются различные положительные стороны исчисления ПО--формул. Выделим те, которые, на наш взгляд, являются наиболее подходящими для задачи построения системы АДТ.

\begin{enumerate}
\item\label{flbs} Любая ПО--формула имеет {\em крупноблочную структуру} и только {\em позитивные кванторы} $\exists$ и $\forall$.
%
\item\label{fsimple} Хотя ПО--формулы содержат как квантор $\exists$ так и квантор $\forall$, структура же ПО--формул {\em простая}, {\em регулярная} и {\em предсказуемая} благодаря регулярному чередованию кванторов $\exists$ и $\forall$ по всем ветвям формулы.
%
\item Нет необходимости в предварительной обработке первоначальной формулы первого порядка с помощью процедуры сколемизации (удаление переменных, связанных кванторами существования. Процедура сколемизации приводит к увеличению сложности термов а значит ив сей формулы. Кроме того, данная особенность делает исчисление более человеко--ориентированным.
%
\item ``Теоретические" кванторы $\forall x$ и $\exists x$ обычно не используются в формализации человеческих знаний, вместо этого чаще используются типовые кванторы $\forall x(A \rightarrow \sqcup)$ и $\exists x(A \& \sqcup)$ \cite{Bourbaki}, \cite{ICDS2000}.
%
\item\label{frule} Исчисление ПО--формул содержит только одно правило вывода и это свойство (как и в случае МР) делает исчисление поболее машинно--ориентированным. Кроме того правило выводя является крупноблочным, что лучше сказывается на человеко--ориентированности.
%
\item\label{froot} Процедура ЛВ фокусируется в ближайшей окрестности корня ПО--формулы, благодаря особенностям \ref{flbs}, \ref{fsimple}.
%
\item\label{fqa} ЛВ может быть представлен в терминах {\em вопросно-ответной} процедуры, а не в технических терминах формального вывода (в терминах логических связок, атомов и др.). Базовый конъюнкт может быть интерпретирован как {\em база фактов}.
%
\item Имеется естественный {\em ИЛИ--параллелизм}.
%
\item\label{fheu} Благодаря \ref{flbs}, \ref{fsimple}, \ref{froot}, \ref{fqa} процедура ЛВ {\em хорошо совместима с конкретными эвристиками приложения}, а также с эвристиками общего управления выводом. Благодаря \ref{frule} доказательство состоит из крупноблочных шагов, и оно хорошо {\em наблюдаемо} и {\em управляемо}.
%
\item Благодаря \ref{fqa}, \ref{fheu} доказательство может быть интерпретировано человеком.
%
\item Семантика исчисления ПО--формул может быть изменена без какой либо модификации аксиом или правила вывода $\bomega$. Такая модификация реализуется просто путём ограничений на применение правила вывода $\bomega$ и позволяет трансформировать классическую семантику исчисления ПО--формул в немонотонную, интуиционистскую, и т.п. Примеры использования таких семантик представлены в \cite{ICDS2000}.
\end{enumerate}


%-------------------------------------------------------------
%------------------------------------------------------------
\section{Постановка задачи}



%%% Local Variables:
%%% mode: latex
%%% TeX-master: "dis"
%%% End:
