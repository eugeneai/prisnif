\chapter{Введение}


%--------------------------------------------
%----------------------об автоматизации рассуждений----------------
%-----------------------------------------------
\section{Об автоматизации рассуждений}
%исторический взгляд
\subsection{Исторический взгляд}
Пионерские идеи об автоматизации (механизации) рассуждений скорее всего высказали Раймунд Луллий (1235-1315) и позднее Готфрид Лейбниц (1646-1716). Луллий описывал некую механическую машину для выведения новых истин [], а Лейбниц предложил создать формальный универсальный язык <<lingua characteristica>>, в котором можно было бы формулировать любые утверждения и создать для него исчисление <<calculus ratiocinator>>. Так исчисление могло быть механизированно решать вопрос об истинности утверждений, и это бы стало <<освобождением человеческого разума от его собственных представлений о вещах>>. В качестве примера Лейбниц рассматривал ситуацию, когда два участника спора, для проверки кто из них прав, переводят свои аргументы на <<lingua characteristica>> и потом говорят: <<Calculemus!>> --- подсчитаем. Хотя эти идеи так и остались идеями не найдя никакого материального воплощения, фактически Лейбницом были сформулированы две основных составляющих для автоматизации рассуждений: специальный язык для записи утверждений, который ныне называют <<формальным>> и правила оперирования выражениями этого языка (правила вывода), в совокупности составляющими формальную систему; наличие механизма способного работать с данным языком, в качестве которого сейчас выступает компьютер.
%Появление последних датировано серединой XX-ого века, но к этому времени уже были сформулированы многие принципы формальных систем.

Первая составляющая развивалась в контексте математической логики и оснований математики. Тут стоит указать на роль работ следующих исследователей: Август де Морган, Джордж Буль, Чарльз Пирс, как основоположники логики высказывания и исследователи в области алгебры; Готтлоб Фреге \cite{Frege}, \cite{Sourcebook}, впервые описавший язык и исчисление предикатов (в несколько неестественной для современного человека форме); Джузеппе Пеано, Бертран Рассел, Давид Гильберт, развили результаты Фреге, и поставили важные задачи \cite{GilbertAkkerman}; Курт Гёдель, Алан Тьюринг, Алонзо Черч, получили результаты, касающиеся пределов возможностей формальных систем, в частности полнота \cite{Godel1929} и неразрешимость теорий первого порядка, неполнота систем, выражающих арифметику; Альберт Туральф Сколем, показал что для данного множества истинных высказываний можно механически найти их доказательство; Жак Эрбран доказал, что для истинного математического предложения можно доказать что оно истинно и предложил метод доказательства. В совокупности с результатом Тьюринга и Черча это говорит о полуразрешимости теорий первого порядка.
Идеи Эрбрана и по сей день лежат в основе многих методов.

Вторая составляющая развивалась в контексте вычислительных машин, основным толчком к созданию которых было в большей степени обусловлено выживанием в условиях второй мировой войны, можно отметить работы Джона фон Неймана, Норберта Винера, Алана Тьюринга, Конрада Цузе.

Развитие аппарата математической логики и вычислительных машин естественно привело к созданию первых работающих на практике систем автоматического доказательства теорем: Программа М. Дэвиса в 1954 г. работающая на компьютере <<Johniac>>, доказала что сумма двух четных чисел есть четное число (первое доказательство математического утверждения, произведенное на компьютере) \cite{LogicComp}; <<Логик-теоретик>>, разработанный А. Невелом, Г. Саймоном, Дж. К. Шоу \cite{Newell1}, \cite{Newell2} в 1956 году для доказательства некоторых задач из Principia Mathematica \cite{PrinMat}, причем данная система была направлена на моделирование человеческих рассуждений; В 1958 г. Ван Хао создаёт систему, доказавшую 350 задач из Principia Mathematica \cite{WangHao}.

Началом сильного развития области АДТ явился метод резолюций \cite{Robinson_1965} предложенный Дж.~Робинсоном в 1965 году (работы велись совместно с Д.~Карсоном и Л.~Уосом). Причиной его успеха явилась достаточно хорошая пригодность формализма для реализации на компьютере, в частности однородность представления данных и единственность правила вывода. Стоит отметить что данный метод и по сей день занимает доминирующее положение среди теоретического базиса для систем АДТ.

Отметим что уже в то время указывалось на важность применения эвристик []. И уже тогда зародилась некоторая конкуренция между чисто машинными подходами и эвристическими. В 1960-ые Л.М.~Нортон разработал эвристический прувер для теории групп [ЛиК5]. [Надо ещё примеров].

В 1994 году системой EQP была доказана открытая математическая проблема, что сильно повысило планку возможностей пруверов.

О современных системах сказано ниже.

Взятые за базис данной работы первопорядковый язык и исчисление позитивно-образованных формул (ПО--формул) разработаны академиком Васильевым С.Н., к.ф-м.н. Жерловым А.К. и академиком Матросовым В.М. на основе языка типово-кванторных формул, использовавшегося для формализации свойств динамических систем и синтеза теорем в методе векторных функций Ляпунова.

%Более подробную историческую справку по рассматриваемому вопросу можно найти в работах \cite{LogicComp}, \cite{SourceBook}.

%Добавить про современное состояние дел. %Верификация программного и аппаратного обеспечения, логические и constraint-языки программирования... Приложения АДТ, в т.ч. для логико--динамических систем (Васильев). Или это все дальше есть?

%Формализм по-формул возник как развитие методов... и был предложен в 80-ые годы[], как уже говорилось характерной его чертой является с одной стороны машинная ориентированность, с другой стороны ориентированность на человека. Заметим, что имеется ввиду не просто ориентировнность на какую-то область знаний (типа теории групп, или геометрии) а в принципе на возможность человека вкладывать свои знания в систему. [криво как-то звучит, но суть ясна]. В [КК] дано введение в формализм по-формул а также описаны варианты использования его в задачах интеллектного управления динамическими системами.


\subsection{Область применения систем АДТ}
Первые системы АДТ предназначались скорее для подтверждения возможности автоматизации рассуждений а также для удовлетворения научного интереса авторов. Чуть позднее с появлением логического программирования, интерес перешел в сферу некоторых прикладных проблем искусственного интеллекта ИИ (однако это уже не совсем АДТ).

На сегодняшней день использование АДТ (с обоснованием его эффективности) замечено в следующих областях: верификация программ [], синтез программ \cite{Butakov1}, верификация оборудования [], обработки естественных языков \cite{ATP_NLP}, решения задачек типа оригами, сокобана \cite{Origami}, в исследовании protocol languages [], исследование безопасности информационных потоков \cite{ATP_Flow}, view deletion в базах данных \cite{ATP_DB}, семантическом вебе [], составлении расписаний [], задачах управления [], и даже компьютерном зрении \cite{ATP_Vision} и др.

Наибольшую популярность имеют: верификация программных и аппаратных систем; синтез программного обеспечения; решение некоторых проблем математики (библиотека TPTP); логическое программирование; дедуктивные базы данных. 

%---------------проблемы----
\subsection{Современные пруверы, проблематика и актуальность}

Мы классифицируем системы автоматизации логического вывода следующим образом:
\begin{enumerate}
\item Классические. Предназначены для автоматического доказательств теорем, выраженных языками первого порядка. Отличительной чертой является множество реализованных методик общего характера для повышения эффективности доказательства. Как правило не предназначены для какой-либо специальной предметной области, и могут рассматриваться как универсальные. Наиболее известные из систем: Otter, Vampire (надо добавить, что вампир может компилировать задачу, но он не использует никакой содержательной информации о задаче, просто структуры данных <<статические>> и проиндексированные), E, SPASS, EQP, Prover9 и др.  В частности  с помощью EQP была доказана открытая математическая проблема \cite{McCuneRob}, а Vampire уже много лет является победителем турнира среди систем АДТ \cite{CASC}. Кроме того, общей чертой данных систем, является использование лишь синтаксической информации о решаемой задаче.

\item Системы, предназначенные для заранее определенного класса задач и неклассических логик. Например для различных алгебраических систем [], геометрические пруверы [] и др. Характерны тем что либо в принципе предназначены только для указанного класса, либо показывают хорошую производительность на задачах такого класса, но при этом могут быть использованы и для решения других задач. Кроме того, к этому классу могут быть причислены системы АДТ для логик высшего порядка или скажем для конструктивных логик или модальных.

\item Настраиваемые (полиморфные) системы. От части к таким системам можно отнести и системы из предыдущих классов, однако под настраиванием мы понимаем в большей степени настройку в соответствии с содержательной информацией о задаче. Как правило, такие системы представляют собой комбинацию существующих систем, например Isabelle \cite{Isabelle} или система предложенная в [problem-oriented application of ATP]. Coq \cite{LaCoq} предлагает использование так называемых <<тактик>>, соединение нескольких типовых шагов вывода в один шаг. Интерактивные системы, хотя и не могут в полной мере быть автоматическими, всё же они интересны тем что используют философию тесного взаимодействия человека и компьютера.
\end{enumerate}
Конечно, некоторые системы могут быть причислены сразу к нескольким классам.


%---------
%Небольшой список с которым мы возможно будет сравниваться: Vampire, Otter, E, SPASS, EQP, Prover9, Isabelle, SNARK, Joke, KeY, ACL2, Coq, NuPRL.
%--------

(Боюсь соврать)
Из существующих систем АДТ для метода ПО--формул, выделим ряд версий системы КВАНТ \cite{dissChe, Che2, QUANT4}, разработанных раннее Е.А. Черкашиным и его учениками, система Бутакова предназначенная для решения задач разбора LL-1 граммаик. Предложенная в данной работе система является качественно новой, поскольку все предыдущие могут быть получены из неё путём спецификации. Система Бутакова конкретно предназначено для синтеза разборщика. Система Черкашина была разработано без учета функциональных символов, неограниченных переменных, работы с предикатом равенства, параллельных схем алгоритмов, стратегий экономии памяти и др. %Были еще 2 версии - Е.Сомова, ориентированная на управление техническими системами, она, фактически, поддерживала очень узкий подкласс ПО--формул. И версия Ш.Б.(забыл уже фамилию А вот я помню! Гулямов он, Шухрата Баратовича. Диссер 1992-3 г.)
%---------

Авторами самых передовых пруверов неоднократно высказывалось что сложность разработки начинает достигать своего предела. Внедряемые методики крайне сложны в реализации, в совмещении с другими методиками, портят расширяемость систему, и вносят абсолютное запутывание в то что происходит внутри системы во время работы. Так, например, в работе \cite{BTPstickel}, автор M. Stickel приводит главу с названием в переводе на русский язык <<Индексирование, необходимое зло>>, при этом сам Стикель является автором одной из очень популярной и применяемой в самых передовых пруверах методики индексироваия путями (path indexing).

В работах \cite{Eprover} указывается на то что их система более интеллектна по сравнению с другими системами. Такая интеллектность обусловлена возможностью более гибкого подключения дополнительных эвристик для решения задач определенного класса.

Отсюда можно сказать что в целом действительно в области АДТ есть необходимость в новых методах либо в развитии старых в том направлении, чтобы привлекать возможности человека а так же гибкости системы для варьирования своего поведения в зависимости от решаемой задачи.

%--------АКТУАЛЬНОСТЬ---------------
%\section{Об актуальности}
Отметим что существуют теоремы, которые имеют сколь угодно большой минимальный вывод, т.е., даже если прувер сумеет перебрать все варианты доказательства до определенной глубины, то он всё равно не докажет теорему лежащую вне этого пространства поиска. Отсюда, исследование любых методов, позволяющих расширить возможное пространство поиска (глубину вывода), является актуальным. С нашей точки зрения исчисление ПО--формул как раз даёт фундаментальное сокращение шагов вывода, в силу крупноблочности правила вывода.

С другой стороны, такое отодвижение границ даёт лишь потенциальную возможность для доказательства более широкого класса формул. Полный перебор с возрастанием глубины вывода отнимет очень много вычислительных ресурсов. Поэтому так же актуальным является и возможная интеллектуализация процедуры поиска вывода, т.е., дополнительные эвристики о задаче, позволяющие двигаться в предположительно верном направлении поиска.

Многие современные системы АДТ разрабатываются уже много лет (есть примеры 30-ти летнего опыта), в них внедрено огромное количество методик. Тем не менее эти методики как правило не носят интеллектный характер, а направлены лишь на эффективную обработку данных (например, индексироваие), сокращение потребляемой памяти и т.д. Либо предложен некоторый ограниченный ряд стратегий общего характера.

Кроме того в силу интеллектности и свойства интерпретируемости вывода в исчислении ПО-формул, было бы интересно разработать методики преобразования формального вывода к виду, пригодному для понимания человеку.



%-------------------------------------------------------------------
%---------------- ПО-формулы--------------------------------------
\section{Теоретический базис разработки}

В основе разрабатываемой системы лежит исчисление ПО--формул (ПОФ).

Исчисление ПОФ $\boldsymbol{JF}$ есть тройка $\left\langle \boldsymbol{LF}, Ax\boldsymbol{JF}, \bomega \right\rangle$, где $\boldsymbol{LF}$ --- язык ПОФ, $Ax\boldsymbol{JF}$ --- единственная схема аксиом и $\bomega$ --- единственное правило вывода $\boldsymbol{JF}.$

%---------------------ЯЗЫК ПО-ФОРМУЛ--------------------
\subsection{Язык позитивно--образованных формул}

Будем обозначать множество всех конъюнктов как $Con$ и положим что {\em конъюнкт} либо конечное множестве обычных атомов языка предикатов первого порядка либо $\boldsymbol{False}$, где $\boldsymbol{False}$ удовлетворяет условию $A \subset \boldsymbol{False} $ для любого $A \in Con$. Пустой конъюнкт обозначается как $\boldsymbol{True}$. Очевидно что если $A \in Con$ тогда $A \cup \boldsymbol{False} = \boldsymbol{False}$. Атомы любого конъюнкта, исключая $\boldsymbol{True}$ и $\boldsymbol{False}$, могут содержать переменные, константные и функциональные символы.

\begin{definition}\label{def:1}
Пусть $\bar{x}$ есть множество переменных и $A$ есть конъюнкт. Правильно-построенные формулы языка ПОФ определяются следующим образом:

Выражение вида $\exists \bar{x}\colon A$ есть $\exists$--формула; выражение вида $\forall \bar{x}\colon A$ есть $\forall$--формула.

Пусть $G_1,\ldots,G_k$ являются $\exists$--формулами, тогда $\forall$--формула имеет следующий вид: $\forall \bar{x}\colon A\left(G_1,\ldots,G_k\right)$.

Пусть $G_1,\ldots,G_k$ являются $\forall$--формулами, тогда $\exists$--формула имеет следующий вид: $\exists \bar{x}\colon A\left(G_1,\ldots,G_k\right)$.

\end{definition}

Выражение является правильно построенной ПОФ если оно построено только по правилам Определения \ref{def:1}.

Переменные из $\bar{x}$ связаны соответствующими кванторами и называются $\forall$-переменные и $\exists$-переменные, соответственно.

$\forall$-переменная которая не встречается в соответствующем конъюнкте называется {\em неограниченной} переменной.

%semantic
Теперь определим семантику ПОФ как семантику соответствующих формул языка предикатов первого порядка.

\begin{definition}\label{def:semantic}
Пусть $A = \{A_1,\ldots,A_l\}$ есть конъюнкт и $\bar{x} = \{x_1,\ldots,x_n\}$ --- множество переменных. Через $A^{\&}$ обозначим $A_1 \&\ldots\&A_l$, при этом $\boldsymbol{False}^{\&}= False, \boldsymbol{True}^{\&}=True$ (пропозициональные константы). Через $F^{\text{\tiny{FOF}}}$ обозначим образ соответствующей ПОФ $F$ в языке FOL.

Если $F= \exists \bar{x}\colon A$ то $F^{\text{\tiny{FOF}}} = \exists x_1 \ldots \exists x_n \left(A^{\&}\right).$

Если $F = \forall \bar{x}\colon A$ то $F^{\text{\tiny{FOF}}} = \forall x_1 \ldots \forall x_n \left(A^{\&}\right).$

Если $F = \exists \bar{x}\colon A\left(G_1,\ldots,G_k\right)$ то $F^{\text{\tiny{FOF}}} = \exists x_1 \ldots \exists x_n  \left(A^{\&} \& \left(G_{1}^{\text{\tiny{FOF}}} \& \ldots \& G_{k}^{\text{\tiny{FOF}}}\right)\right).$

Если $F = \forall \bar{x}\colon A\left(G_1,\ldots,G_k\right)$ то $F^{\text{\tiny{FOF}}} = \forall x_1 \ldots \forall x_n \left(A^{\&}\rightarrow \left(G_{1}^{\text{\tiny{FOF}}} \vee\ldots\vee G_{k}^{\text{\tiny{FOF}}}\right)\right).$

\end{definition}

Любая ПОФ очевидно имеет структуру дерева. Таким образом, для удобства читаемости мы будем представлять их как древовидные структуры а также пользоваться соответствующей терминологии: узел, корень, ветвь, лист и т.д.

Если ПОФ $F$ начинается с $\forall\colon\boldsymbol{True}$ узла, и каждый лист $F$ является $\exists$--узлом, то $F$ называется ПОФ в {\em канонической форме}.
Очевидно что любая ПОФ $F$ может быть приведена к каноническому виду с помощью следующих преобразований:
\begin{enumerate}
\item Если $F$ не каноническая $\forall$--формула, тогда $\forall\colon \boldsymbol{True}\left(\exists\colon \boldsymbol{True}\left(F\right)\right)$ есть ПОФ начинающаяся с $\forall\colon\boldsymbol{True}$.
\item Если $F$ есть $\exists$--формула, тогда $\forall\colon \boldsymbol{True}\left(F\right)$ есть ПОФ начинающаяся с $\forall\colon\boldsymbol{True}$.
\item Если $F$ имеет лист $\forall \bar{x}\colon A$, тогда новый узел $\exists\colon\boldsymbol{False}$ может быть добавлен как потомок.
\end{enumerate}

В дальнейшем, если не оговорено обратного, будем рассматривать только ПОФы в каноническом виде.

Некоторые части ПОФ имеются специальные названия: корневой (0-глубины) узел называется {\em корнем} ПО--формулы; любой узел глубины 1 называется {\em базой} ПО--формулы; максимальное поддерево начинающееся с узла глубины 1 называется {\em базовой подформулой}; любой узел глубины 2 называется {\em вопросом} к базе; максимальное поддерево начинающееся с узла глубины 2 называется {\em подформулой-вопросом}; максимальное поддерево начинающееся с узла глубины 3 называется {\em консеквентом}. В соответствии с определением семантики если узел чётной (нечетной) глубины имеет более чем одного потомка, то будем говорить что этот узел имеет {\em дизъюнктивное ветвление} (соответственно {\em конъюнктивное ветвление}).

\begin{example}
Рассмотрим формулу языка FOL
$$F= \neg\bigl(\forall x\:\exists y P(x,y)\rightarrow \exists z P(z,z)\bigr).$$
Образ $F^{\text{\tiny{PCF}}}$ формулы $F$ в языке ПОФ есть
$$F^{\text{\tiny{PCF}}} = \forall\colon \boldsymbol{True}-\exists\colon\boldsymbol{True} \left\{
\begin{array}{lcl}
 \forall x\colon\boldsymbol{True} & - & \exists y\colon P(x,y) \\
 \forall z\colon P(z,z) & - & \exists\colon\boldsymbol{False}
\end{array}
\right.$$

\end{example}


%-------------------------ИСЧИСЛЕНИЕ ПО-ФОРМУЛ-------------------
\subsection{Исчисление позитивно-образованных формул}

Схема аксиом исчисления ПОФ $\boldsymbol{JF}$ имеет следующую форму:
$$ Ax\boldsymbol{JF} = \forall\colon\boldsymbol{True}\left(\exists \bar{x}_1\colon\boldsymbol{False}\left(\widetilde{\Phi}_1\right),\ldots,\exists \bar{x}_n\colon\boldsymbol{False}\left(\widetilde{\Phi}_n\right)\right) $$

В исчислении ПОФ для того что бы доказать $F$ мы будем пытаться опровергнуть её отрицание, поэтому аксиом\app{а} исчисления ПОФ ---  тождественно ложное высказывание. Таким образом процесс вывода в исчислении ПОФ является процессом {\em опровержения}.

\begin{definition}
\label{ircond}
Будем говорить что вопрос $\forall \bar{y}\colon A$ к базе $\exists \bar{x}\colon B$ имеет {\em ответ} $\theta$  тогда и только тогда, когда $\theta$ есть подстановка $\bar{y} \rightarrow H^{\infty}$ и $A\theta \subseteq B$, где $H^{\infty}$ есть Эрбранов универсум основанный на $\exists$--переменных из $\bar{x}$, константных и функциональных символах которые встречаются в соответствующей базе.
\end{definition}

%The unique unary inference rule $\bomega$ is defined as follows.

\begin{definition}
\label{omega}
Если $F$ имеет структуру $\forall\colon\boldsymbol{True}\left(\exists \bar{x}\colon B\left(\Phi\right),\Sigma\right)$, где $\Sigma$ есть список других базовых подформул, и $\Phi$ есть список подформул--вопросов содержащий подформулу--вопрос $\forall \bar{y}\colon A(\exists \bar{z}_i\colon C_i\left(\Psi_i\right))_{i=\overline{1,k}}$, тогда заключение $\bomega F$ есть результат применения унарного правила вывода $\bomega$ к вопросу $\forall \bar{y}\colon A$ с ответом $\theta$, и $\bomega F = \forall\colon\boldsymbol{True}(\exists \bar{x} \cup \bar{z}_i\colon B \cup C_i\theta\left(\Phi \cup \Psi_i\theta\right)_{i=\overline{1,k}},\Sigma).$

\end{definition}

После соответствующего переименования некоторых переменных в каждой подформуле, выражение $\bomega F$ будет удовлетворять всем условиям правильно--построенных ПОФ.

%
Любая конечная последовательность ПОФ $F, \bomega F, \bomega^2 F,\ldots,\bomega^n F$, где $\bomega^n F \in Ax\boldsymbol{JF}$ называется {\em опровержением} $F$ в исчислении ПОФ. Иногда будем использовать слово вывод вместо опровержение.

Вопрос с консеквентом $\exists:\boldsymbol{False}$ называется {\em целевым вопросом}. Если вопрос имеет дизъюнктивное ветвление, то ответ на этот вопрос приводит к расщеплению соответствующей базовой подформулы на несколько новых.

Для ясности рассмотрим пример.

\begin{example}[Опровержение в $\boldsymbol{JF}$]\label{proofexample}


\begin{equation*}\label{ex:f1}
	F_1 = \forall\colon\boldsymbol{True} - \exists\colon S(e)(Q_1,Q_2,Q_3,Q_4);
\end{equation*}
\begin{equation*}
	\begin{array}{l}
	Q_1 = \forall x\colon S(x) - \exists\colon A(a) \\
	Q_2 = \forall x,y\colon C(x),D(y) - \exists\colon\boldsymbol{False} \\
	Q_3 = \forall x,y\colon B(x),C(f(y)) - \exists\colon\boldsymbol{False} \\
	Q_4 =
	\forall x\colon A(x) -
	\left\lbrace
	\begin{array}{l}
		\exists y\colon B(y),C(f(x)) \\
		\exists \colon C(x) - \forall z\colon A(z),C(z) - \exists\colon D(f(z))
	\end{array}\right.
	\end{array}
\end{equation*}
На первом шаге вывода существует только один ответ $\{x \rightarrow e\}$ на вопрос $Q_1$. После применения $\bomega$ с этим ответом, формула приобретает следующий вид:
\begin{equation*}\label{ex:f2}
	F_2 = \forall\colon\boldsymbol{True} - \exists\colon S(e),A(a)(Q_1,Q_2,Q_3,Q_4)
\end{equation*}
На втором шаге вывода существует только один ответ $\{x \rightarrow a\}$ на вопрос $Q_4$. После применения $\bomega$ с этим ответом, формула расщепляется, потому что $Q_4$ имеет дизъюнктивное ветвление. И теперь формулы имеет следующий вид:

\begin{equation*}\label{ex:f3}
F_3 =
\forall:\boldsymbol{True} -
\left\lbrace
\begin{array}{l}
	\exists y_1\colon S(e),A(a),B(y_1),C(f(a)) -
	\left\lbrace
	\begin{array}{l}
		Q_1 \\ \cdots \\ Q_4
	\end{array}\right. \\
	\exists\colon S(e)A(a),C(a) -
	\left\lbrace
	\begin{array}{l}
		Q_1 \\ \cdots \\ Q_4 \\
		\forall z\colon A(z),C(z) - \exists\colon D(f(z))
	\end{array}\right. \\
\end{array}\right.
\end{equation*}

На третьем шаге вывода первая база может быть опровергнута ответом $\{x \rightarrow y_1; y \rightarrow a\}$ на целевой вопрос $Q_3$. Опровергнутая база (подформула) для удобства представления может быть удалена из списка базовых подформул.

На четвертом шаге вывода существует ответ $\{z \rightarrow a\}$ на пятый новый вопрос. И формула приобретает следующий вид:
\begin{equation*}\label{ex:f5}
	F_4 = \forall\colon\boldsymbol{True} - \exists\colon S(e),A(a), C(a),D(f(a))
	\left\lbrace
	\begin{array}{l}
		Q_1 \\ \cdots \\ Q_4 \\
		\forall z\colon A(z),C(z) - \exists\colon D(f(z))
	\end{array}\right.
\end{equation*}

На пятом шаге вывода единственная база может быть опровергнута ответом $\{x \rightarrow a; y \rightarrow f(a)\}$ на целевой вопрос $Q_4$.

Опровержение закончено поскольку все базы опровергнуты.

\end{example}

Корректность и полнота доказаны в \cite{}.

%----------------------------------ОСОБЕННОСТИ ПО-ФОРМУЛ--------------------
\subsection{Особенности ПО-формул}
В литературе выделяются различные положительные стороны исчисления ПО--формул. Выделим те, которые, на наш взгляд, являются наиболее подходящими для задачи построения системы АДТ.

\begin{enumerate}
\item\label{flbs} Любая ПО--формула имеет {\em крупноблочную структуру} и только {\em позитивные кванторы} $\exists$ и $\forall$.
%
\item\label{fsimple} Хотя ПО--формулы содержат как квантор $\exists$ так и квантор $\forall$, структура же ПО--формул {\em простая}, {\em регулярная} и {\em предсказуемая} благодаря регулярному чередованию кванторов $\exists$ и $\forall$ по всем ветвям формулы.
%
\item Нет необходимости в предварительной обработке первоначальной формулы первого порядка с помощью процедуры сколемизации (удаление переменных, \app{связанных кванторами существования}. Процедура сколемизации приводит к увеличению сложности термов а значит ив сей формулы. Кроме того, данная особенность делает исчисление более человеко--ориентированным.
%
\item ``Теоретические" кванторы $\forall x$ и $\exists x$ обычно не используются в формализации человеческих знаний, вместо этого чаще используются типовые кванторы $\forall x(A \rightarrow \sqcup)$ и $\exists x(A \& \sqcup)$ \cite{Bourbaki}, \cite{ICDS2000}.
%
\item\label{frule} Исчисление ПО--формул содержит только одно правило вывода и это свойство (как и в случае МР) делает исчисление поболее машинно--ориентированным. Кроме того правило выводя является крупноблочным, что лучше сказывается на человеко--ориентированности.
%
\item\label{froot} Процедура ЛВ фокусируется в ближайшей окрестности корня ПО--формулы, благодаря особенностям \ref{flbs}, \ref{fsimple}.
%
\item\label{fqa} ЛВ может быть представлен в терминах {\em вопросно-ответной} процедуры, а не в технических терминах формального вывода (в терминах логических связок, атомов и др.). Базовый конъюнкт может быть интерпретирован как {\em база фактов}.
%
\item Имеется естественный {\em ИЛИ--параллелизм}.
%
\item\label{fheu} Благодаря \ref{flbs}, \ref{fsimple}, \ref{froot}, \ref{fqa} процедура ЛВ {\em хорошо совместима с конкретными эвристиками приложения}, а также с эвристиками общего управления выводом. Благодаря \ref{frule} доказательство состоит из крупноблочных шагов, и оно хорошо {\em наблюдаемо} и {\em управляемо}.
%
\item Благодаря \ref{fqa}, \ref{fheu} доказательство может быть интерпретировано человеком. Такая интерпретируемость доказательства довольно важна с точки зрения человеко--машинных приложений. Таким образом, как говорилось выше, исчисление ПО--формул не только машинно--ориентированное, но и \rem{человеко--ориентированное}{Уже 4-е(?) повторение}.
%
\item Семантика исчисления ПО--формул может быть изменена без какой либо модификации аксиом или правила вывода $\bomega$. Такая модификация реализуется просто путём ограничений на применение правила вывода $\bomega$ и позволяет трансформировать классическую семантику исчисления ПО--формул в немонотонную, интуиционистскую, и т.п. Примеры использования таких семантик представлены в \cite{ICDS2000}.
\end{enumerate}


%-------------------------------------------------------------
%------------------------------------------------------------
\section{Формальная информация о диссертации}

% Объект — это процесс или явление, порождающее проблемную ситуацию и взятое исследователем для изучения. Предмет — это то, что находится в рамках, в границах объекта. Объект — это та часть научного знания, с которой исследователь имеет дело. Предмет исследования — это тот аспект проблемы, исследуя который, мы познаем целостный объект, выделяя его главные, наиболее существенные признаки. Предмет диссертационного исследования чаще всего совпадает с определением его темы или очень близок к нему. Объект и предмет исследования как научные категории соотносятся как общее и частное.

%----------------
\paragraph{Объектом исследования}\hspace{-1em} является разработка эффективных алгоритмов для автоматического доказательства теорем в исчислении ПО--формул. В работе производится повышение эффективности АДТ по следующим критериям: сокращение времени решения задач; \app{экономное использование оперативной памяти}; минимизация количества шагов логического вывода; увеличение ширины класса решаемых задач. %Или это таки предмет уже???

%----------------
\paragraph{Предметом исследования}\hspace{-1em} являются методы повышения производительности логического вывода, оригинально разработанные или адаптированные с других систем АДТ в новой системе АДТ.

%-----------------??
\paragraph{Методика исследования.}\hspace{-1em} Использование исчисления ПО-формул как базиса для АДТ; Использование языка D для программирования системы; Анализ других систем АДТ на предмет применения их методов.

%-----------------??
\paragraph{Цель диссертационной работы:}\hspace{-1em} Разработка высокопроизводительной человеко-- машинно--ориентированной программной системы для автоматического доказательства теорем в исчислении позитивно--образованных формул.

%-----------------??
\paragraph{Основные задачи диссертационной работы.} Для достижения описанной выше цели решаются следующие задачи:
\begin{enumerate}
\item Исследование языка и исчисления ПО--формул, анализ его свойств, влияющих на возможность адаптации существующих алгоритмов и разработке новых;
\item Разработка эффективных структур данных представления ПО--формул в памяти компьютера;
\item Разработка алгоритмов преобразования ПО--формул для организации автоматического логического ввода;
\item Адаптация существующих\app{методик реализации алгоритмов} АДТ для исчисления ПО--формул;
\item Разрешение основных проблем исчисления ПО--формул, негативно влияющих на эффективность автоматического поиска ЛВ;
\item Исследование вопросов автоматизации построения эффективного логического вывода с использованием предиката равенства;
\item Разработка программной системы АДТ, создание инструментальных средств для программирования специальных версий АДТ, ориентированных на определенные классы теоретических и практических задач поиска ЛВ.
\item Реализация программной инфраструктуры, обеспечивающей человеку возможность управления процессом поиска логического вывода;
\item Апробация разработанных программных средств в решении тестовых задач.
\end{enumerate}

%-------------------------------------------------------------------
%------НАУЧНАЯ НОВИЗНА, ПРАКТИЧЕСКАЯ ЗНАЧИМОСТЬ, АПРОБАЦИЯ----
%------------------------------------------------------
%-----??
\paragraph{Научная новизна, практическая значимость и апробация полученных результатов}
Разработаны новые и адаптированы существующие алгоритмы, реализующие метод АДТ для реализации исчислений ПО--формул. Использование данных алгоритмов позволило реализовать полноценную программную систему АДТ для этих исчислений. В результате проведенного исследования получены следующие новые научные результаты:
\begin{enumerate}
\item Изучены свойства исчисления ПО--формул, определяющие характер адаптации существующих алгоритмов;
\item Предложены и реализованы ряд стратегий поиска логического вывода ПО--формул с неограниченными переменными;
\item Предложена и реализована стратегия $k,m$--ограничения;
\item Усовершенствован подход к представлению структур данных ПО--формул в направлении использования разделения оперативной памяти;
\item Адаптированы алгоритмы индексирования термов для системы АДТ ПО-формул;
\item Предложены и реализованы стратегии параллельного логического вывода для системы АДТ ПО--формул;
\item Предложен подход для работы с предикатом равенства, без прямого использования аксиом равенства.
\end{enumerate}

%--------------------??
\paragraph{Практическая значимость}\hspace{-1em} представляется следующими основными результатами:
\begin{enumerate}
\item Разработана система АДТ и инструментальная среда разработки систем АДТ, направленных на решения специальных классов задач.
\item Выделены классы задач, на которых разработанная система ведёт себя более эффективно чем самые производительные современные системы АДТ, \app{предложены специальные стратегии поиска ЛВ для этих классов};
\item Налажено взаимодействие с TPTP; % Это совсем частная задача.
\item Расширена область применения систем АДТ (Леса?).
%Надо, чтобы корреляция была с поставленными задачами.
\end{enumerate}


Кроме того, немаловажным результатом является возможность использования системы АДТ базирующейся на раннее не использованном исчислении ПО--формул (новый взгляд на задачи).

Работы и исследования, проведенные в рамках  диссертации, выполнены в ФГБОУ ВПО <<Иркутский государственный университет>> и в Институте динамики систем и теории управления СО РАН.

Исследования поддержаны также грантами:
\begin{enumerate}
\item РФФИ 08-07-98005-р\_сибирь\_а <<Программные технологии логико-математического моделирования динамики лесных ресурсов Байкальского региона>>;
\item Программа <<Университетский кластер>>;
\item Федеральная целевая программа «Научные и научно-педагогические кадры инновационной России» на 2009-2013 годы, госконтракт № П696;
\item MIPRO канает?
\end{enumerate}

\paragraph{Результаты, выносимые на защиту}\hspace{-1em} представляется следующими основными результатами:

% Где внедрены разработанные инструментальные средства.
Разработанные инструментальные средства внедрены в учебный процесс вузов города Иркутска, в частности, в Институте математики, экономики и информатики Иркутского государственного университета (ИМЭИ ИГУ). Разработана новый вариант курса ``Технологии разработки программного обеспечения'' на кафедре информационных технологий ИМЭИ ИГУ.

% ??? внедрение на производстве и в фирмах

Инструментальные средства также внедрены...

% ??? Собственное применение.



%------------------------------------------
\paragraph{Представление работы.} Материалы работы докладывались на
\begin{itemize}
\item Международной конференции <<Мальцевские чтения>>, г.Новосибирск, 24-28 августа 2009 г.;
\item Семинаре ИДСТУ СО РАН <<Ляпуновские чтения>>, ИДСТУ СО РАН, г. Иркутск, 21-23 декабря 2009 г.;
\item Всероссийской конференции молодых ученых <<Математическое моделирование и информационные технологии>>, г. Иркутск, 15-21 марта 2010 г.;
\item Международной конференции <<Облачные вычисления. Образование. Исследования. Разработки>>, г.Москва 15-16 апреля 2010 г.;
\item Международном симпозиуме по компьютерным наукам в России. Семинар <<Семантика, спецификация и верификация программ: теория и приложения>>, г.Казань, 14-15 июня 2010 г.;
\item 4-ой Всероссийской конференции <<Винеровские чтения>>, г.Иркутск, 9-14 марта 2011г.
\item 34-ом международном симпозиуме <<MIPRO>>, г.Опатия, Хорватия, 23-27 мая 2011г.
\item 4-ой Всероссийской мультиконференции по проблемам управления, с. Дивноморское, 3-8 октября 2011г.
\item 13-ой национальной конференции по искусственному интеллекту с международным участием (КИИ-2012), г. Белгород, 16-20 октября 2012 г.
\end{itemize}

%-------------------------------
\paragraph{Публикации по теме диссертации.} По теме диссертации опубликовано 3 работы в журналах из перечня рецензируемых научных журналов и изданий ВАК, в которых публикуются научные результаты диссертации на соискание ученой степени доктора и кандидата наук, и Web of Science:
\begin{enumerate}
\item Давыдов А.В., Ларионов А.А., Черкашин Е.А. Об исчислении
позитивно-образованных формул для автоматического доказательства
теорем. // Моделирование и анализ информационных систем. 2010.T. 17, N
4, С. 60--69.
\item Ларионов А.А., Черкашин Е.А., Терехин И.Н. Системные предикаты для
управления логическим выводом в системе автоматического доказательства
теорем для исчисления позитивно-образованых формул. //Вестник
Бурятского государственного университета, 2011, выпуск 9. Серия
Математика. Информатика. с. 94-98.
\item Ларионов А.А., Черкашин Е.А. Параллельные схемы алгоритмов
автоматического доказательства теорем в исчислении
позитивно-образованных формул. // Дистанционное и виртуальное
обучение. Февраль 2012. No. 2, С. 93-100.
\item Davydov A.V., Larionov A.A., Cherkashin E.A. On the calculus of
positively constructed formulas for automated theorem proving. //
Automatic Control and Computer Sciences (AC\&CS). N 1. 2011.
\end{enumerate}

%---------------------------------
\paragraph{Личный вклад автора.} Все представленные результаты получены лично автором или в соавторстве с научным руководителем Черкашиным~Е.А. и Давыдовым~А.В. Автором лично разработаны:
\begin{enumerate}
\item Механизмы логического вывода в исчислении ПО-формул (структуры данных и их обработка);
\item БОЛЬШЕ НИЧЕГО ЛОЛОЛО
\end{enumerate}

Из печатных работ, опубликованных диссертантом в соавторстве, в текст глав 2,3 и 4 диссертации вошли только те результаты, которые содержат непосредственный определяющий творческий вклад автора диссертации на этапах проектирования и разработки программного обеспечения. В перечисленных публикациях все результаты, связанные с вопросами реализации и использования программной системы принадлежат автору.

%-------------------------------------
\paragraph{Соответствие паспорту специальности}
05.13.11 -- Математическое и программное обеспечение вычислительных машин, комплексов и компьютерных сетей. Область исследования 5. Программные системы символьных вычислений; 8. Параллельное программирование.

%-------------------------------
\paragraph{Структура и объем диссертации.} Диссертация состоит из четырех глав, первая из которых --- вводная, заключения, списка литературы и приложения. Основной текст изложен на \pageref{pg:main} страницах машинописного текста, полный объем диссертации \pageref{pg:total} страниц. В работе содержится 20 рисунков. Список литературы содержит 50 наименований.

Глава 2 посвящена стратегическому базису программной системы. Выделены некоторые проблемы ПО--исчисления, решение которых требовалось. Представлен ряд стратегий, адаптированных из существующих систем АДТ и новые стратегии для исчисления ПО-формул, рассмотрен вопрос решения задач с предикатом равенства.

Глава 3 посвящена аспектам реализации и использования системы. Системные предикаты, работа с арифметикой, трансляторы.

В главе 4 представлены методы и результаты тестирования системы. В частности дано описание библиотеки TPTP. Решена задача о лесах с применением внешних эвристик.

В заключении приводится список результатов, полученных в диссертации. % выносимых на защиту.

Приложение содержит описание грамматики входного языка, пример протокола вывода для решения задач, сводную информацию о решенных задач из библиотеки TPTP.

%--------благодарности-------------
\paragraph{Благодарности.} Автор благодарит к.т.н. Черкашина~Е.А. за руководство диссертационной работой и помощь в подготовке рукописи, Давыдова~А.В. за ценные указания в работе.


%%% Local Variables:
%%% mode: latex
%%% TeX-master: "dis"
%%% End:
